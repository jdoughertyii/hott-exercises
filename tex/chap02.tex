\begin{coqdoccode}
\coqdocemptyline
\end{coqdoccode}
\section{Homotopy type theory}



 \exerdone{2.1}{103}  
Show that the three obvious proofs of Lemma 2.1.2 are pairwise equal.
 

 \soln
Lemma 2.1.2 states that for every type $A$ and every $x, y, z : A$, there is a
function
\[
  (x = y) \to (y = z) \to (x = z)
\]
written $p \mapsto q \mapsto p \ct q$ such that $\refl{x} \ct \refl{x} =
\refl{x}$ for all $x : A$.  Each proof is an object $\ct_{i}$ of type
\[
  \ct_{i} : \prd{x,y,z:A} (x=y)\to(y=z)\to(x=z)
\]
So we need to show that $\ctu = \ctd = \ctt$.


The first proof is induction over $p$.  Consider the family
\[
  C_{1}(x, y, p) \defeq 
  \prd{z:A} (y=x) (x=z)
\]
we have
\[
\lam{z}{q}q :
\left(\prd{z:A} (x=z) \to (x=z)\right)
  \equiv
  C_{1}(x, x, \refl{x})
\]
So by path induction, there is a function
\[
  p \ctu q : (x=z)
\]
such that $\refl{x} \ctu q \equiv q$. 


For the shorter version, we say that by induction it suffices to consider the
case where $y$ is $x$ and $p$ is $\refl{x}$.  Then given some $q : x = z$, we
want to construct an element of $x=z$; but this is just $q$, so induction gives
us a function $p \ctu q : x = z$ such that $\refl{x} \ctu q \defeq q$.
\begin{coqdoccode}
\coqdocemptyline
\coqdocnoindent
\coqdockw{Definition} \coqdef{chap02.cat'}{cat'}{\coqdocdefinition{cat'}} \{\coqdocvar{A} : \coqdockw{Type}\} \{\coqdocvar{x} \coqdocvar{y} \coqdocvar{z} : \coqdocvariable{A}\} (\coqdocvar{p} : \coqdocvariable{x} \coqdocnotation{=} \coqdocvariable{y}) (\coqdocvar{q} : \coqdocvariable{y} \coqdocnotation{=} \coqdocvariable{z}) : \coqdocvariable{x} \coqdocnotation{=} \coqdocvariable{z}.\coqdoceol
\coqdocindent{1.00em}
\coqdoctac{induction} \coqdocvar{p}. \coqdoctac{apply} \coqdocvar{q}.\coqdoceol
\coqdocnoindent
\coqdockw{Defined}.\coqdoceol
\coqdocemptyline
\end{coqdoccode}
For the second, consider the family
\[
  C_{2}(y, z, q) \defeq 
  \prd{z:A} (x=y) \to (x=z)
\]
and element
\[
  \lam{z}{p}p
  :
  \left(
    \prd{z:A} (x=z)\to (x=z)
  \right)
  \equiv
  C_{2}(z,z,\refl{z})
\]
Induction gives us a function
\[
  p \ctd q : (x = z)
\]
such that
\[
  p \ctd \refl{z} = \refl{z}
\] \begin{coqdoccode}
\coqdocnoindent
\coqdockw{Definition} \coqdef{chap02.cat''}{cat'{}'}{\coqdocdefinition{cat'{}'}} \{\coqdocvar{A}:\coqdockw{Type}\} \{\coqdocvar{x} \coqdocvar{y} \coqdocvar{z} : \coqdocvariable{A}\} (\coqdocvar{p} : \coqdocvariable{x} \coqdocnotation{=} \coqdocvariable{y}) (\coqdocvar{q} : \coqdocvariable{y} \coqdocnotation{=} \coqdocvariable{z}) : \coqdocvariable{x} \coqdocnotation{=} \coqdocvariable{z}.\coqdoceol
\coqdocindent{1.00em}
\coqdoctac{induction} \coqdocvar{q}. \coqdoctac{apply} \coqdocvar{p}.\coqdoceol
\coqdocnoindent
\coqdockw{Defined}.\coqdoceol
\coqdocemptyline
\end{coqdoccode}


Finally, for $\ctt$, we have the construction from the text.  Take the type
families
\[
  D(x,y,p) \defeq 
  \prd{z:A}(y=z) \to  (x=z)
\]
and
\[
  E(x, z, q) \defeq (x = z)
\]
Since $E(x,x,\refl{x}) \equiv (x=x)$, we have $e(x) \defeq \refl{x} :
E(x,x,\refl{x})$, and induction gives us a function
\[
  d : \left(\prd{x,z:A}\prd{q:x=z}(x=z)\right) 
  \equiv
  \prd{x:A}D(x, x, \refl{x})
\]
So path induction again gives us a function
\[
  f : \prd{x,y,z:A}(x=y) \to(y=z) \to (x=z)
\]
Which we can write $p \ctt q : (x=z)$.  By the definitional equality of $f$, we
have that $\refl{x} \ct q \equiv d(x)$, and by the definitional equality of
$d$, we have $\refl{x} \ct \refl{x} \equiv \refl{x}$. \begin{coqdoccode}
\coqdocemptyline
\coqdocnoindent
\coqdockw{Definition} \coqdef{chap02.cat'''}{cat'{}'{}'}{\coqdocdefinition{cat'{}'{}'}} \{\coqdocvar{A}:\coqdockw{Type}\} \{\coqdocvar{x} \coqdocvar{y} \coqdocvar{z} : \coqdocvariable{A}\} (\coqdocvar{p} : \coqdocvariable{x} \coqdocnotation{=} \coqdocvariable{y}) (\coqdocvar{q} : \coqdocvariable{y} \coqdocnotation{=} \coqdocvariable{z}) : \coqdocvariable{x} \coqdocnotation{=} \coqdocvariable{z}.\coqdoceol
\coqdocindent{1.00em}
\coqdoctac{induction} \coqdocvar{p}, \coqdocvar{q}. \coqdoctac{reflexivity}.\coqdoceol
\coqdocnoindent
\coqdockw{Defined}.\coqdoceol
\coqdocemptyline
\end{coqdoccode}
Now, to show that $p \ctu q = p \ctd q = p \ctt q$, which we will do by
induction on $p$ and $q$.  For the first pair, we want to construct for every
$x, y, z : A$, $p : x = y$, and $q : y = z$, an element of $p \ctu q = p \ctd
q$.  By induction on $p$, it suffices to assume that $y$ is $x$ and $p$ is
$\refl{x}$; similarly, by induction on $q$ it suffices to assume that $z$ is
also $x$ and $q$ is $\refl{x}$.  Then by the computation rule for $\ctu$,
$\refl{x} \ctu \refl{x} \equiv \refl{x}$, and by the computation rule for
$\ctd$, $\refl{x} \ctd \refl{x} \equiv \refl{x}$.  Thus we have
\[
  \refl{\refl{x}} : (\refl{x} \ctu \refl{x} = \refl{x} \ctd \refl{x})
\]
which provides the necessary data for induction.


Writing this out a bit more fully for practice, we have the family
\[
  C : \prd{x, y : A} (x = y) \to \UU
\]
defined by
\[
  C(x, y, p) \defeq 
  \prd{z:A}\prd{q:y=z}(p \ctu q = p \ctd q)
\] 
and in order to apply induction, we need an element of
\[
  \prd{x:A}C(x, x, \refl{x}) 
  \equiv
  \prd{x, z:A}\prd{q:x=z}(\refl{x} \ctu q = \refl{x} \ctd q)
  \equiv
  \prd{x, z:A}\prd{q:x=z}(q = \refl{x} \ctd q)
\]
Define $D(x, z, q) \defeq (q = \refl{x} \ctd q)$.  Then
\[
  \refl{\refl{x}} : 
  D(x, x, \refl{x}) 
  \equiv
  (\refl{x} = \refl{x} \ctd \refl{x})
  \equiv
  (\refl{x} = \refl{x})
\]
So by induction we have a function $f:\prd{x, z: A}\prd{p:x=z}(q = \refl{x}
\ctd q)$ with $f(x, x, \refl{x}) \defeq \refl{\refl{x}}$.  Thus we have the
element required for induction on $p$, and there is a function 
\[
  f' : \prd{x, y, z:A}\prd{p:x=y}\prd{q:y=z}(p \ctu q = p \ctd q)
\]
which we wanted to show.  \begin{coqdoccode}
\coqdocemptyline
\coqdocnoindent
\coqdockw{Theorem} \coqdef{chap02.cat' eq cat''}{cat'\_eq\_cat'{}'}{\coqdoclemma{cat'\_eq\_cat'{}'}} : \coqdockw{\ensuremath{\forall}} \{\coqdocvar{A}:\coqdockw{Type}\} \{\coqdocvar{x} \coqdocvar{y} \coqdocvar{z} : \coqdocvariable{A}\} (\coqdocvar{p} : \coqdocvariable{x} \coqdocnotation{=} \coqdocvariable{y}) (\coqdocvar{q} : \coqdocvariable{y} \coqdocnotation{=} \coqdocvariable{z}), \coqdoceol
\coqdocindent{1.00em}
\coqdocnotation{(}\coqref{chap02.cat'}{\coqdocdefinition{cat'}} \coqdocvariable{p} \coqdocvariable{q}\coqdocnotation{)} \coqdocnotation{=} \coqdocnotation{(}\coqref{chap02.cat''}{\coqdocdefinition{cat'{}'}} \coqdocvariable{p} \coqdocvariable{q}\coqdocnotation{)}.\coqdoceol
\coqdocnoindent
\coqdockw{Proof}.\coqdoceol
\coqdocindent{1.00em}
\coqdoctac{induction} \coqdocvar{p}, \coqdocvar{q}. \coqdoctac{reflexivity}.\coqdoceol
\coqdocnoindent
\coqdockw{Defined}.\coqdoceol
\coqdocemptyline
\end{coqdoccode}
For the next pair, we again use induction.  For all $x, y, z : A$,
$p : x = y$, and $q : y = z$, we need to construct an element of $p
\ctd q = p \ctt q$.  By induction on $p$ and $q$, it suffices to consider the
case where $y$ and $z$ are $x$ and $p$ and $q$ are $\refl{x}$.  Then
$(\refl{x} \ctd \refl{x} = \refl{x} \ctt \refl{x}) \equiv (\refl{x} =
\refl{x})$, and $\refl{\refl{x}}$ inhabits this type.
 \begin{coqdoccode}
\coqdocemptyline
\coqdocnoindent
\coqdockw{Theorem} \coqdef{chap02.cat'' eq cat'''}{cat'{}'\_eq\_cat'{}'{}'}{\coqdoclemma{cat'{}'\_eq\_cat'{}'{}'}} : \coqdockw{\ensuremath{\forall}} \{\coqdocvar{A}:\coqdockw{Type}\} \{\coqdocvar{x} \coqdocvar{y} \coqdocvar{z} : \coqdocvariable{A}\} (\coqdocvar{p} : \coqdocvariable{x} \coqdocnotation{=} \coqdocvariable{y}) (\coqdocvar{q} : \coqdocvariable{y} \coqdocnotation{=} \coqdocvariable{z}), \coqdoceol
\coqdocindent{1.00em}
\coqdocnotation{(}\coqref{chap02.cat''}{\coqdocdefinition{cat'{}'}} \coqdocvariable{p} \coqdocvariable{q}\coqdocnotation{)} \coqdocnotation{=} \coqdocnotation{(}\coqref{chap02.cat'''}{\coqdocdefinition{cat'{}'{}'}} \coqdocvariable{p} \coqdocvariable{q}\coqdocnotation{)}.\coqdoceol
\coqdocnoindent
\coqdockw{Proof}.\coqdoceol
\coqdocindent{1.00em}
\coqdoctac{induction} \coqdocvar{p}, \coqdocvar{q}. \coqdoctac{reflexivity}.\coqdoceol
\coqdocnoindent
\coqdockw{Defined}.\coqdoceol
\coqdocemptyline
\end{coqdoccode}
The third proof goes exactly the same. \begin{coqdoccode}
\coqdocemptyline
\coqdocnoindent
\coqdockw{Theorem} \coqdef{chap02.cat' eq cat'''}{cat'\_eq\_cat'{}'{}'}{\coqdoclemma{cat'\_eq\_cat'{}'{}'}} : \coqdockw{\ensuremath{\forall}} \{\coqdocvar{A}:\coqdockw{Type}\} \{\coqdocvar{x} \coqdocvar{y} \coqdocvar{z} : \coqdocvariable{A}\} (\coqdocvar{p} : \coqdocvariable{x} \coqdocnotation{=} \coqdocvariable{y}) (\coqdocvar{q} : \coqdocvariable{y} \coqdocnotation{=} \coqdocvariable{z}), \coqdoceol
\coqdocindent{1.00em}
\coqdocnotation{(}\coqref{chap02.cat'}{\coqdocdefinition{cat'}} \coqdocvariable{p} \coqdocvariable{q}\coqdocnotation{)} \coqdocnotation{=} \coqdocnotation{(}\coqref{chap02.cat'''}{\coqdocdefinition{cat'{}'{}'}} \coqdocvariable{p} \coqdocvariable{q}\coqdocnotation{)}.\coqdoceol
\coqdocnoindent
\coqdockw{Proof}.\coqdoceol
\coqdocindent{1.00em}
\coqdoctac{induction} \coqdocvar{p}, \coqdocvar{q}. \coqdoctac{reflexivity}.\coqdoceol
\coqdocnoindent
\coqdockw{Defined}.\coqdoceol
\coqdocemptyline
\end{coqdoccode}
\noindent
Note that all three of these proofs must end with \coqdockw{Defined} instead of \coqdockw{Qed} if
we want to make use of the computational identity (e.g., $p \ctu \refl{x}
\equiv p$)
that they produce, as we will in the next exercise.
 

 \exerdone{2.2}{103}  
Show that the three equalities of proofs constructed in the previous exercise form a commutative triangle.  In other words, if the three definitions of concatenation are denoted by $(p \ctu q)$, $(p \ctd q)$, and $(p \ctt q)$, then the concatenated equality
\[
  (p \ctu q) = (p \ctd q) = (p \ctt q)
\]
is equal to the equality $(p \ctu q) = (p \ctt q)$.
 

 \soln
Let $x, y, z : A$, $p : x = y$, $q : y = z$, and let $r_{12} : (p \ctu q = p
\ctd q)$, $r_{23} : (p \ctd q = p \ctt q)$, and $r_{13} : (p \ctu q = p
\ctt q)$ be the proofs from the last exercise.  We want to show that $r_{12}
\ct r_{23} = r_{13}$, where $\ct = \ct_{3}$ is the concatenation operation from
the book.  By induction on $p$ and $q$, it suffices to consider the case where
$y$ and $z$ are $x$ and $p$ and $q$ are $\refl{x}$.  Then we have $r_{12}
\equiv \refl{\refl{x}}$, $r_{23} \equiv \refl{\refl{x}}$, and
$r_{13} \equiv \refl{\refl{x}}$ by the definitions.  But then the type we're
trying to witness is
\[
    (\refl{\refl{x}} \ct \refl{\refl{x}} = \refl{\refl{x}})
    \equiv
    (\refl{\refl{x}} = \refl{\refl{x}})
\]
from the definition of $\ct$, so $\refl{\refl{\refl{x}}}$ is our witness.
 \begin{coqdoccode}
\coqdocemptyline
\coqdocnoindent
\coqdockw{Theorem} \coqdef{chap02.comm triangle}{comm\_triangle}{\coqdoclemma{comm\_triangle}} : \coqdockw{\ensuremath{\forall}} (\coqdocvar{A}:\coqdockw{Type}) (\coqdocvar{x} \coqdocvar{y} \coqdocvar{z} : \coqdocvariable{A}) (\coqdocvar{p} : \coqdocvariable{x} \coqdocnotation{=} \coqdocvariable{y}) (\coqdocvar{q} : \coqdocvariable{y} \coqdocnotation{=} \coqdocvariable{z}),\coqdoceol
\coqdocindent{1.00em}
\coqdocnotation{(}\coqref{chap02.cat' eq cat''}{\coqdoclemma{cat'\_eq\_cat'{}'}} \coqdocvariable{p} \coqdocvariable{q}\coqdocnotation{)} \coqdocnotation{@} \coqdocnotation{(}\coqref{chap02.cat'' eq cat'''}{\coqdoclemma{cat'{}'\_eq\_cat'{}'{}'}} \coqdocvariable{p} \coqdocvariable{q}\coqdocnotation{)} \coqdocnotation{=} \coqdocnotation{(}\coqref{chap02.cat' eq cat'''}{\coqdoclemma{cat'\_eq\_cat'{}'{}'}} \coqdocvariable{p} \coqdocvariable{q}\coqdocnotation{)}.\coqdoceol
\coqdocnoindent
\coqdockw{Proof}.\coqdoceol
\coqdocindent{1.00em}
\coqdoctac{induction} \coqdocvar{p}, \coqdocvar{q}. \coqdoctac{reflexivity}.\coqdoceol
\coqdocnoindent
\coqdockw{Qed}.\coqdoceol
\coqdocemptyline
\end{coqdoccode}
\exerdone{2.3}{103}  
Give a fourth, different proof of Lemma 2.1.2, and prove that it is equal to
the others.
 

 \soln
Let $x, y : A$ and $p : x = y$.  Rather than fixing some $q$ and constructing an element of $x = z$ out of that, we can directly construct an element of
\[
  \prd{z:A} (y = z) \to (x = z)
\]
by induction on $p$.  It suffices to consider the case where $y$ is
$x$ and $p$ is a $\refl{x}$, which then makes it easy to produce such
an element; namely,
\[
  \lam{z}\idfunc{x=z} : \prd{z:A} (x = z) \to (x = z)
\]
Induction then gives us a function $p \ctq q : (x = z)$ such that
$\lam{q}(\refl{x} \ctq q) \defeq \idfunc{x=z}$.
 \begin{coqdoccode}
\coqdocemptyline
\coqdocnoindent
\coqdockw{Definition} \coqdef{chap02.cat''''}{cat'{}'{}'{}'}{\coqdocdefinition{cat'{}'{}'{}'}} \{\coqdocvar{A}:\coqdockw{Type}\} \{\coqdocvar{x} \coqdocvar{y} \coqdocvar{z} : \coqdocvariable{A}\} (\coqdocvar{p} : \coqdocvariable{x} \coqdocnotation{=} \coqdocvariable{y}) (\coqdocvar{q} : \coqdocvariable{y} \coqdocnotation{=} \coqdocvariable{z}) : \coqdocvariable{x} \coqdocnotation{=} \coqdocvariable{z}.\coqdoceol
\coqdocindent{1.00em}
\coqdoctac{generalize} \coqdocvar{q}. \coqdoctac{generalize} \coqdocvar{z}.\coqdoceol
\coqdocindent{1.00em}
\coqdoctac{induction} \coqdocvar{p}. \coqdoctac{trivial}.\coqdoceol
\coqdocnoindent
\coqdockw{Defined}.\coqdoceol
\coqdocemptyline
\end{coqdoccode}
To prove that it's equal to the others, we can just show that it's equal to $\ct$ and then use concatenation.  Again by induction on $p$ and $q$, it suffices to consider the case where $y$ and $z$ are $x$ and $p$ and $q$ are $\refl{x}$.  Then we have
\[
  \left((\refl{x} \ctt \refl{x}) = (\refl{x} \ctq \refl{x})\right)
  \equiv
  \left(\refl{x} = \idfunc{x=x}(\refl{x})\right)
  \equiv
  \left(\refl{x} = \refl{x}\right)
\] 
So $\refl{\refl{x}}$ is again our witness.\begin{coqdoccode}
\coqdocemptyline
\coqdocnoindent
\coqdockw{Theorem} \coqdef{chap02.cat''' eq cat''''}{cat'{}'{}'\_eq\_cat'{}'{}'{}'}{\coqdoclemma{cat'{}'{}'\_eq\_cat'{}'{}'{}'}} : \coqdockw{\ensuremath{\forall}} \{\coqdocvar{A}:\coqdockw{Type}\} \{\coqdocvar{x} \coqdocvar{y} \coqdocvar{z} : \coqdocvariable{A}\} (\coqdocvar{p} : \coqdocvariable{x} \coqdocnotation{=} \coqdocvariable{y}) (\coqdocvar{q} : \coqdocvariable{y} \coqdocnotation{=} \coqdocvariable{z}),\coqdoceol
\coqdocindent{1.00em}
\coqdocnotation{(}\coqref{chap02.cat'''}{\coqdocdefinition{cat'{}'{}'}} \coqdocvariable{p} \coqdocvariable{q}\coqdocnotation{)} \coqdocnotation{=} \coqdocnotation{(}\coqref{chap02.cat''''}{\coqdocdefinition{cat'{}'{}'{}'}} \coqdocvariable{p} \coqdocvariable{q}\coqdocnotation{)}.\coqdoceol
\coqdocnoindent
\coqdockw{Proof}.\coqdoceol
\coqdocindent{1.00em}
\coqdoctac{induction} \coqdocvar{p}, \coqdocvar{q}. \coqdoctac{reflexivity}.\coqdoceol
\coqdocnoindent
\coqdockw{Qed}.\coqdoceol
\coqdocemptyline
\end{coqdoccode}
\exerdone{2.4}{103} 
Define, by induction on $n$, a general notion of $n$-dimensional path
in a type $A$, simultaneously with the type of boundaries for such paths. 

 \soln
A $0$-path in $A$ is an element $x : A$, so the type of $0$-paths is just $A$.
If $p$ and $q$ are $n$-paths, then so is $p = q$.  In the other
direction, the boundary of a $0$-path is empty, and the boundary of an
$n+1$ path is an $n$-path.
\begin{coqdoccode}
\coqdocemptyline
\coqdocnoindent
\coqdockw{Fixpoint} \coqdef{chap02.npath}{npath}{\coqdocdefinition{npath}} (\coqdocvar{A}:\coqdockw{Type}) (\coqdocvar{n}:\coqexternalref{nat}{http://coq.inria.fr/distrib/8.4pl3/stdlib/Coq.Init.Datatypes}{\coqdocinductive{nat}}) : \coqdockw{Type} :=\coqdoceol
\coqdocindent{1.00em}
\coqdockw{match} \coqdocvariable{n} \coqdockw{with}\coqdoceol
\coqdocindent{2.00em}
\ensuremath{|} \coqexternalref{O}{http://coq.inria.fr/distrib/8.4pl3/stdlib/Coq.Init.Datatypes}{\coqdocconstructor{O}} \ensuremath{\Rightarrow} \coqdocvariable{A}\coqdoceol
\coqdocindent{2.00em}
\ensuremath{|} \coqexternalref{S}{http://coq.inria.fr/distrib/8.4pl3/stdlib/Coq.Init.Datatypes}{\coqdocconstructor{S}} \coqdocvar{n'} \ensuremath{\Rightarrow} \coqexternalref{:type scope:'x7B' x ':' x 'x26' x 'x7D'}{http://coq.inria.fr/distrib/8.4pl3/stdlib/Coq.Init.Specif}{\coqdocnotation{\{}}\coqdocvar{p} \coqexternalref{:type scope:'x7B' x ':' x 'x26' x 'x7D'}{http://coq.inria.fr/distrib/8.4pl3/stdlib/Coq.Init.Specif}{\coqdocnotation{:}} \coqexternalref{:type scope:'x7B' x ':' x 'x26' x 'x7D'}{http://coq.inria.fr/distrib/8.4pl3/stdlib/Coq.Init.Specif}{\coqdocnotation{(}}\coqref{chap02.boundary}{\coqdocdefinition{boundary}} \coqdocvariable{A} \coqdocvar{n'}\coqexternalref{:type scope:'x7B' x ':' x 'x26' x 'x7D'}{http://coq.inria.fr/distrib/8.4pl3/stdlib/Coq.Init.Specif}{\coqdocnotation{)}} \coqexternalref{:type scope:'x7B' x ':' x 'x26' x 'x7D'}{http://coq.inria.fr/distrib/8.4pl3/stdlib/Coq.Init.Specif}{\coqdocnotation{\&}} \coqexternalref{:type scope:'x7B' x ':' x 'x26' x 'x7D'}{http://coq.inria.fr/distrib/8.4pl3/stdlib/Coq.Init.Specif}{\coqdocnotation{\{}}\coqdocvar{q} \coqexternalref{:type scope:'x7B' x ':' x 'x26' x 'x7D'}{http://coq.inria.fr/distrib/8.4pl3/stdlib/Coq.Init.Specif}{\coqdocnotation{:}} \coqexternalref{:type scope:'x7B' x ':' x 'x26' x 'x7D'}{http://coq.inria.fr/distrib/8.4pl3/stdlib/Coq.Init.Specif}{\coqdocnotation{(}}\coqref{chap02.boundary}{\coqdocdefinition{boundary}} \coqdocvariable{A} \coqdocvar{n'}\coqexternalref{:type scope:'x7B' x ':' x 'x26' x 'x7D'}{http://coq.inria.fr/distrib/8.4pl3/stdlib/Coq.Init.Specif}{\coqdocnotation{)}} \coqexternalref{:type scope:'x7B' x ':' x 'x26' x 'x7D'}{http://coq.inria.fr/distrib/8.4pl3/stdlib/Coq.Init.Specif}{\coqdocnotation{\&}} \coqdocvar{p} \coqdocnotation{=} \coqdocvar{q}\coqexternalref{:type scope:'x7B' x ':' x 'x26' x 'x7D'}{http://coq.inria.fr/distrib/8.4pl3/stdlib/Coq.Init.Specif}{\coqdocnotation{\}\}}}\coqdoceol
\coqdocindent{1.00em}
\coqdockw{end}\coqdoceol
\coqdocnoindent
\coqdoceol
\coqdocnoindent
\coqdockw{with} \coqdef{chap02.boundary}{boundary}{\coqdocdefinition{boundary}} (\coqdocvar{A}:\coqdockw{Type}) (\coqdocvar{n}:\coqexternalref{nat}{http://coq.inria.fr/distrib/8.4pl3/stdlib/Coq.Init.Datatypes}{\coqdocinductive{nat}}) : \coqdockw{Type} :=\coqdoceol
\coqdocindent{1.00em}
\coqdockw{match} \coqdocvariable{n} \coqdockw{with}\coqdoceol
\coqdocindent{2.00em}
\ensuremath{|} \coqexternalref{O}{http://coq.inria.fr/distrib/8.4pl3/stdlib/Coq.Init.Datatypes}{\coqdocconstructor{O}} \ensuremath{\Rightarrow} \coqdocinductive{Empty}\coqdoceol
\coqdocindent{2.00em}
\ensuremath{|} \coqexternalref{S}{http://coq.inria.fr/distrib/8.4pl3/stdlib/Coq.Init.Datatypes}{\coqdocconstructor{S}} \coqdocvar{n'} \ensuremath{\Rightarrow} (\coqref{chap02.npath}{\coqdocdefinition{npath}} \coqdocvariable{A} \coqdocvar{n'})\coqdoceol
\coqdocindent{1.00em}
\coqdockw{end}.\coqdoceol
\coqdocemptyline
\end{coqdoccode}
\exerdone{2.5}{103} 
Prove that the functions
\begin{align*}
 (f(x) = f(y)) &\to (p_{*}(f(x)) = f(y)) \qquad\qquad\text{and} \\
 (p_{*}(f(x)) = f(y)) &\to (f(x) = f(y))
\end{align*}
are inverse equivalences. 

 \soln 
I take it that ``inverse equivalences'' means that each of the maps is the
quasi-inverse of the other.  Suppose that $x, y : A$, $p : x = y$, and $f : A
\to B$.  Then we have the objects 
\[
  \mapfunc{f}(p) : (f(x) = f(y)) \\
  \transconst{B}{p}{f(x)} : (p_{*}(f(x)) = f(x))
\]
thus
\begin{align*}
  \left(\transconst{B}{p}{f(x)} \ct -\right)
  &:
  (f(x)=f(y))
  \to
  (p_{*}(f(x)) = f(y))
  \\
  \left((\transconst{B}{p}{f(x)})^{-1} \ct -\right)
  &:
  (p_{*}(f(x))=f(y))
  \to
  (f(x) = f(y))
\end{align*}
Which are our maps.  Composing the first with the second, we obtain an element
\[
  \left(\transconst{B}{p}{f(x)} \ct
  \left((\transconst{B}{p}{f(x)})^{-1} \ct -\right)\right)
\]
of $f(x) = f(y)$.  Using Lemma 2.1.4, we can show that this is homotopic to the
identity:
\begin{align*}
  &\prd{q:f(x)=f(y)}
  \left(
  \transconst{B}{p}{f(x)} \ct
  \left((\transconst{B}{p}{f(x)})^{-1} \ct q\right)
  =
  q\right)
  \\&= 
  \prd{q:f(x)=f(y)}
  \left(
  \left(\transconst{B}{p}{f(x)} \ct
  (\transconst{B}{p}{f(x)})^{-1}\right) \ct q
  =
  q\right)
  \\&= 
  \prd{q:f(x)=f(y)}
  \left(
  q
  =
  q\right)
\end{align*}
which is inhabited by $\refl{q}$.  The same argument goes the other way, so
this concatenation is an equivalence.
\begin{coqdoccode}
\coqdocemptyline
\coqdocnoindent
\coqdockw{Definition} \coqdef{chap02.eq2 3 6}{eq2\_3\_6}{\coqdocdefinition{eq2\_3\_6}} \{\coqdocvar{A} \coqdocvar{B} : \coqdockw{Type}\} \{\coqdocvar{x} \coqdocvar{y} : \coqdocvariable{A}\} (\coqdocvar{f} : \coqdocvariable{A} \coqexternalref{:type scope:x '->' x}{http://coq.inria.fr/distrib/8.4pl3/stdlib/Coq.Init.Logic}{\coqdocnotation{\ensuremath{\rightarrow}}} \coqdocvariable{B}) (\coqdocvar{p} : \coqdocvariable{x} \coqdocnotation{=} \coqdocvariable{y}) :\coqdoceol
\coqdocindent{1.00em}
\coqexternalref{:type scope:x '->' x}{http://coq.inria.fr/distrib/8.4pl3/stdlib/Coq.Init.Logic}{\coqdocnotation{(}}\coqdocvariable{f} \coqdocvariable{x} \coqdocnotation{=} \coqdocvariable{f} \coqdocvariable{y}\coqexternalref{:type scope:x '->' x}{http://coq.inria.fr/distrib/8.4pl3/stdlib/Coq.Init.Logic}{\coqdocnotation{)}} \coqexternalref{:type scope:x '->' x}{http://coq.inria.fr/distrib/8.4pl3/stdlib/Coq.Init.Logic}{\coqdocnotation{\ensuremath{\rightarrow}}} \coqexternalref{:type scope:x '->' x}{http://coq.inria.fr/distrib/8.4pl3/stdlib/Coq.Init.Logic}{\coqdocnotation{(}}@\coqdocdefinition{transport} \coqdocvar{\_} (\coqdockw{fun} \coqdocvar{\_} \ensuremath{\Rightarrow} \coqdocvariable{B}) \coqdocvar{\_} \coqdocvar{\_} \coqdocvariable{p} (\coqdocvariable{f} \coqdocvariable{x}) \coqdocnotation{=} \coqdocvariable{f} \coqdocvariable{y}\coqexternalref{:type scope:x '->' x}{http://coq.inria.fr/distrib/8.4pl3/stdlib/Coq.Init.Logic}{\coqdocnotation{)}} :=\coqdoceol
\coqdocindent{1.00em}
\coqdockw{fun} \coqdocvar{q} \ensuremath{\Rightarrow} \coqdocnotation{(}\coqdocdefinition{transport\_const} \coqdocvariable{p} (\coqdocvariable{f} \coqdocvariable{x})\coqdocnotation{)} \coqdocnotation{@} \coqdocvariable{q}.\coqdoceol
\coqdocnoindent
\coqdockw{Definition} \coqdef{chap02.eq2 3 7}{eq2\_3\_7}{\coqdocdefinition{eq2\_3\_7}} \{\coqdocvar{A} \coqdocvar{B} : \coqdockw{Type}\} \{\coqdocvar{x} \coqdocvar{y} : \coqdocvariable{A}\} (\coqdocvar{f} : \coqdocvariable{A} \coqexternalref{:type scope:x '->' x}{http://coq.inria.fr/distrib/8.4pl3/stdlib/Coq.Init.Logic}{\coqdocnotation{\ensuremath{\rightarrow}}} \coqdocvariable{B}) (\coqdocvar{p} : \coqdocvariable{x} \coqdocnotation{=} \coqdocvariable{y}) :\coqdoceol
\coqdocindent{1.00em}
\coqexternalref{:type scope:x '->' x}{http://coq.inria.fr/distrib/8.4pl3/stdlib/Coq.Init.Logic}{\coqdocnotation{(}}@\coqdocdefinition{transport} \coqdocvar{\_} (\coqdockw{fun} \coqdocvar{\_} \ensuremath{\Rightarrow} \coqdocvariable{B}) \coqdocvar{\_} \coqdocvar{\_} \coqdocvariable{p} (\coqdocvariable{f} \coqdocvariable{x}) \coqdocnotation{=} \coqdocvariable{f} \coqdocvariable{y}\coqexternalref{:type scope:x '->' x}{http://coq.inria.fr/distrib/8.4pl3/stdlib/Coq.Init.Logic}{\coqdocnotation{)}} \coqexternalref{:type scope:x '->' x}{http://coq.inria.fr/distrib/8.4pl3/stdlib/Coq.Init.Logic}{\coqdocnotation{\ensuremath{\rightarrow}}} \coqexternalref{:type scope:x '->' x}{http://coq.inria.fr/distrib/8.4pl3/stdlib/Coq.Init.Logic}{\coqdocnotation{(}}\coqdocvariable{f} \coqdocvariable{x} \coqdocnotation{=} \coqdocvariable{f} \coqdocvariable{y}\coqexternalref{:type scope:x '->' x}{http://coq.inria.fr/distrib/8.4pl3/stdlib/Coq.Init.Logic}{\coqdocnotation{)}} :=\coqdoceol
\coqdocindent{1.00em}
\coqdockw{fun} \coqdocvar{q} \ensuremath{\Rightarrow} \coqdocnotation{(}\coqdocdefinition{transport\_const} \coqdocvariable{p} (\coqdocvariable{f} \coqdocvariable{x})\coqdocnotation{)\^{}} \coqdocnotation{@} \coqdocvariable{q}.\coqdoceol
\coqdocemptyline
\coqdocnoindent
\coqdockw{Definition} \coqdef{chap02.alpha2 5}{alpha2\_5}{\coqdocdefinition{alpha2\_5}} : \coqdockw{\ensuremath{\forall}} \{\coqdocvar{A} \coqdocvar{B}:\coqdockw{Type}\} \{\coqdocvar{x} \coqdocvar{y} : \coqdocvariable{A}\} (\coqdocvar{f}: \coqdocvariable{A} \coqexternalref{:type scope:x '->' x}{http://coq.inria.fr/distrib/8.4pl3/stdlib/Coq.Init.Logic}{\coqdocnotation{\ensuremath{\rightarrow}}} \coqdocvariable{B}) (\coqdocvar{p}:\coqdocvariable{x}\coqdocnotation{=}\coqdocvariable{y}) \coqdocvar{q}, \coqdoceol
\coqdocindent{1.00em}
\coqdocnotation{(}\coqref{chap02.eq2 3 6}{\coqdocdefinition{eq2\_3\_6}} \coqdocvariable{f} \coqdocvariable{p} (\coqref{chap02.eq2 3 7}{\coqdocdefinition{eq2\_3\_7}} \coqdocvariable{f} \coqdocvariable{p} \coqdocvariable{q})\coqdocnotation{)} \coqdocnotation{=} \coqdocvariable{q}.\coqdoceol
\coqdocindent{1.00em}
\coqdoctac{unfold} \coqref{chap02.eq2 3 6}{\coqdocdefinition{eq2\_3\_6}}, \coqref{chap02.eq2 3 7}{\coqdocdefinition{eq2\_3\_7}}. \coqdocvar{path\_induction}. \coqdoctac{reflexivity}.\coqdoceol
\coqdocnoindent
\coqdockw{Defined}.\coqdoceol
\coqdocemptyline
\coqdocnoindent
\coqdockw{Definition} \coqdef{chap02.beta2 5}{beta2\_5}{\coqdocdefinition{beta2\_5}} : \coqdockw{\ensuremath{\forall}} \{\coqdocvar{A} \coqdocvar{B}:\coqdockw{Type}\} \{\coqdocvar{x} \coqdocvar{y} : \coqdocvariable{A}\} (\coqdocvar{f}: \coqdocvariable{A} \coqexternalref{:type scope:x '->' x}{http://coq.inria.fr/distrib/8.4pl3/stdlib/Coq.Init.Logic}{\coqdocnotation{\ensuremath{\rightarrow}}} \coqdocvariable{B}) (\coqdocvar{p}:\coqdocvariable{x}\coqdocnotation{=}\coqdocvariable{y}) \coqdocvar{q}, \coqdoceol
\coqdocindent{1.00em}
\coqdocnotation{(}\coqref{chap02.eq2 3 7}{\coqdocdefinition{eq2\_3\_7}} \coqdocvariable{f} \coqdocvariable{p} (\coqref{chap02.eq2 3 6}{\coqdocdefinition{eq2\_3\_6}} \coqdocvariable{f} \coqdocvariable{p} \coqdocvariable{q})\coqdocnotation{)} \coqdocnotation{=} \coqdocvariable{q}.\coqdoceol
\coqdocindent{1.00em}
\coqdoctac{unfold} \coqref{chap02.eq2 3 6}{\coqdocdefinition{eq2\_3\_6}}, \coqref{chap02.eq2 3 7}{\coqdocdefinition{eq2\_3\_7}}. \coqdocvar{path\_induction}. \coqdoctac{reflexivity}.\coqdoceol
\coqdocnoindent
\coqdockw{Defined}.\coqdoceol
\coqdocemptyline
\coqdocnoindent
\coqdockw{Lemma} \coqdef{chap02.isequiv transportconst}{isequiv\_transportconst}{\coqdoclemma{isequiv\_transportconst}} (\coqdocvar{A} \coqdocvar{B}:\coqdockw{Type}) (\coqdocvar{x} \coqdocvar{y} \coqdocvar{z} : \coqdocvariable{A}) (\coqdocvar{f} : \coqdocvariable{A} \coqexternalref{:type scope:x '->' x}{http://coq.inria.fr/distrib/8.4pl3/stdlib/Coq.Init.Logic}{\coqdocnotation{\ensuremath{\rightarrow}}} \coqdocvariable{B}) (\coqdocvar{p} : \coqdocvariable{x} \coqdocnotation{=} \coqdocvariable{y}) : \coqdoceol
\coqdocindent{1.00em}
\coqdocclass{IsEquiv} (\coqref{chap02.eq2 3 6}{\coqdocdefinition{eq2\_3\_6}} \coqdocvariable{f} \coqdocvariable{p}).\coqdoceol
\coqdocnoindent
\coqdockw{Proof}.\coqdoceol
\coqdocindent{1.00em}
\coqdoctac{apply} (\coqdocdefinition{isequiv\_adjointify} \coqdocvar{\_} (\coqref{chap02.eq2 3 7}{\coqdocdefinition{eq2\_3\_7}} \coqdocvar{f} \coqdocvar{p}) (\coqref{chap02.alpha2 5}{\coqdocdefinition{alpha2\_5}} \coqdocvar{f} \coqdocvar{p}) (\coqref{chap02.beta2 5}{\coqdocdefinition{beta2\_5}} \coqdocvar{f} \coqdocvar{p})).\coqdoceol
\coqdocnoindent
\coqdockw{Qed}.\coqdoceol
\coqdocemptyline
\end{coqdoccode}
\exerdone{2.6}{103} 
Prove that if $p : x = y$, then the function $(p \ct -) : (y = z) \to (x = z)$
is an equivalence.


 \soln
Suppose that $p : x = y$.  To show that $(p \ct -)$ is an equivalence, we need
to exhibit a quasi-inverse to it.  This is a triple $(g, \alpha, \beta)$ of a
function $g:(x = z) \to (y = z)$ and homotopies $\alpha : (p \ct -) \circ g \sim
\idfunc{x=z}$ and $\beta : g \circ (p \ct -) \sim \idfunc{y=z}$.  For $g$, we
can take $(p^{-1} \ct -)$.  For the homotopies, we can use the results of Lemma
2.1.4.  So we have
\[
((p \ct -) \circ g) \sim \idfunc{x=z}
\equiv
\prd{q:x=z}(p \ct (p^{-1} \ct q) = q)
=
\prd{q:x=z}((p \ct p^{-1}) \ct q = q)
=
\prd{q:x=z}(\refl{x} \ct q = q)
=
\prd{q:x=z}(q = q)
\]
which is inhabited by $\refl{q}$ and
\[
(g \circ (p \ct -)) \sim \idfunc{y=z}
\equiv
\prd{q:y=z}(p^{-1} \ct (p \ct q) = q)
=
\prd{q:y=z}((p^{-1} \ct p) \ct q = q)
=
\prd{q:y=z}(\refl{y} \ct q = q)
=
\prd{q:y=z}(q = q)
\]
which is inhabited by $\refl{q}$.  So $(p \ct -)$ has a quasi-inverse, hence it
is an equivalence.
\begin{coqdoccode}
\coqdocemptyline
\coqdocnoindent
\coqdockw{Definition} \coqdef{chap02.alpha2 6}{alpha2\_6}{\coqdocdefinition{alpha2\_6}} \{\coqdocvar{A}:\coqdockw{Type}\} \{\coqdocvar{x} \coqdocvar{y} \coqdocvar{z}:\coqdocvariable{A}\} (\coqdocvar{p}:\coqdocvariable{x}\coqdocnotation{=}\coqdocvariable{y}) (\coqdocvar{q}:\coqdocvariable{x}\coqdocnotation{=}\coqdocvariable{z}) : \coqdocvariable{p} \coqdocnotation{@} \coqdocnotation{(}\coqdocvariable{p}\coqdocnotation{\^{}} \coqdocnotation{@} \coqdocvariable{q}\coqdocnotation{)} \coqdocnotation{=} \coqdocvariable{q}.\coqdoceol
\coqdocindent{1.00em}
\coqdocvar{path\_induction}. \coqdoctac{reflexivity}.\coqdoceol
\coqdocnoindent
\coqdockw{Defined}.\coqdoceol
\coqdocemptyline
\coqdocnoindent
\coqdockw{Definition} \coqdef{chap02.beta2 6}{beta2\_6}{\coqdocdefinition{beta2\_6}} \{\coqdocvar{A}:\coqdockw{Type}\} \{\coqdocvar{x} \coqdocvar{y} \coqdocvar{z}:\coqdocvariable{A}\} (\coqdocvar{p}:\coqdocvariable{x}\coqdocnotation{=}\coqdocvariable{y}) (\coqdocvar{q}:\coqdocvariable{y}\coqdocnotation{=}\coqdocvariable{z}) : \coqdocvariable{p}\coqdocnotation{\^{}} \coqdocnotation{@} \coqdocnotation{(}\coqdocvariable{p} \coqdocnotation{@} \coqdocvariable{q}\coqdocnotation{)} \coqdocnotation{=} \coqdocvariable{q}.\coqdoceol
\coqdocindent{1.00em}
\coqdocvar{path\_induction}. \coqdoctac{reflexivity}.\coqdoceol
\coqdocnoindent
\coqdockw{Defined}.\coqdoceol
\coqdocemptyline
\coqdocnoindent
\coqdockw{Lemma} \coqdef{chap02.isequiv eqcat}{isequiv\_eqcat}{\coqdoclemma{isequiv\_eqcat}} (\coqdocvar{A}:\coqdockw{Type}) (\coqdocvar{x} \coqdocvar{y} \coqdocvar{z} : \coqdocvariable{A}) (\coqdocvar{p} : \coqdocvariable{x} \coqdocnotation{=} \coqdocvariable{y}) : \coqdocclass{IsEquiv} (\coqdockw{fun} \coqdocvar{q}:(\coqdocvariable{y}\coqdocnotation{=}\coqdocvariable{z}) \ensuremath{\Rightarrow} \coqdocvariable{p} \coqdocnotation{@} \coqdocvariable{q}).\coqdoceol
\coqdocnoindent
\coqdockw{Proof}.\coqdoceol
\coqdocindent{1.00em}
\coqdoctac{apply} (\coqdocdefinition{isequiv\_adjointify} \coqdocvar{\_} (\coqdockw{fun} \coqdocvar{q}:(\coqdocvar{x}\coqdocnotation{=}\coqdocvar{z}) \ensuremath{\Rightarrow} \coqdocvar{p}\coqdocnotation{\^{}} \coqdocnotation{@} \coqdocvariable{q}) (\coqref{chap02.alpha2 6}{\coqdocdefinition{alpha2\_6}} \coqdocvar{p}) (\coqref{chap02.beta2 6}{\coqdocdefinition{beta2\_6}} \coqdocvar{p})).\coqdoceol
\coqdocnoindent
\coqdockw{Qed}.\coqdoceol
\coqdocemptyline
\end{coqdoccode}
\exer{2.7}{104} 
State and prove a generalization of Theorem 2.6.5 from cartesian products to
$\Sigma$-types.


 \soln
Suppose that we have types $A$ and $A'$ and type families $B:A\to\UU$ and
$B':A'\to\UU$, along with a function $g:A \to A'$ and a dependent function
$h:\prd{x:A} B(x) \to B'(f(x))$.  We can then define a function $f :
(\sm{x:A}B(x)) \to (\sm{x:A'}B'(x))$ by $f(x) \defeq (g(\fst x), h(\fst x, \snd x))$.


Suppose that $x, y : \sm{a:A}B(a)$, $p : \fst x = \fst y$, and $q : p_{*}(\snd
x) = \snd y$.  Part of the statement of Theorem 2.6.5 is easily generalized: we
are looking to show that
\[
  f(\pair^{=}(p, q)) =_{f(x) = f(y)} \pair^{=}(g(p), -)
\]
for some object
\[
  - : (g(p))_{*}(h(\fst x, \snd x)) =_{B'(g(\fst x))} h(\fst y, \snd y).
\]
The difficulty is finding this object.


 \exer{2.8}{104} 
State and prove an analogue of Theorem 2.6.5 for coproducts.


 \exer{2.9}{104} 
Prove that coproducts have the expected universal property,
\[
  (A + B \to X) \eqvsym (A \to X) \times (B \to X).
\]
Can you generalize this to an equivalence involving dependent functions?


 \soln
To define the forward map, let $h : A+B \to X$ and define $f : (A+B\to X) \to
(A\to X) \times (B \to X)$ by
\[
  f(h) \defeq (\lam{a}h(\inl(a)), \lam{b}h(\inr(b)))
\]
To show that $f$ is an equivalence, we'll need a quasi-inverse, given by
\[
 g(h) \defeq \rec{A+B}(X, \fst h, \snd h)
\]
As well as the homotopies $\alpha : f\circ g \sim \idfunc{(A\to X)\times(B \to
X)}$ and $\beta : g \circ f \sim \idfunc{A+B\to X}$.  For $\alpha$ we need a
witness to
\begin{align*}
  &\prd{h:(A\to X) \times (B\to X)} (f(g(h)) 
  = \idfunc{(A\to X) \times (B\to X)}(h))
  \\&\equiv
  \prd{h:(A\to X) \times (B\to X)} (
  (\lam{a}\rec{A+B}(X, \fst h, \snd h, \inl(a)),
   \lam{b}\rec{A+B}(X, \fst h, \snd h, \inr(b)))
  = h)
  \\&\equiv
  \prd{h:(A\to X) \times (B\to X)} ((\fst h, \snd h) = h)
\end{align*}
and this is inhabited by $\uppt$.  For $\beta$, we need an inhabitant of
\begin{align*}
  &\prd{h:A+B\to X} (g(f(h)) = \idfunc{A+B\to X}(h))
  \\&\equiv
  \prd{h:A+B\to X} (
  \rec{A+B}(X, \lam{a}h(\inl(a)), \lam{b}h(\inr(b)))
  = h)
\end{align*}
which, assuming function extensionality, is inhabited.  So $(g,
\alpha, \beta)$ is a quasi-inverse to $f$, giving the universal property.
\begin{coqdoccode}
\coqdocemptyline
\coqdocnoindent
\coqdockw{Definition} \coqdef{chap02.forward}{forward}{\coqdocdefinition{forward}} \{\coqdocvar{A} \coqdocvar{B} \coqdocvar{X} : \coqdockw{Type}\} (\coqdocvar{h} : (\coqdocvariable{A}\coqexternalref{:type scope:x '+' x}{http://coq.inria.fr/distrib/8.4pl3/stdlib/Coq.Init.Datatypes}{\coqdocnotation{+}}\coqdocvariable{B} \coqexternalref{:type scope:x '->' x}{http://coq.inria.fr/distrib/8.4pl3/stdlib/Coq.Init.Logic}{\coqdocnotation{\ensuremath{\rightarrow}}} \coqdocvariable{X})) : \coqexternalref{:type scope:x '*' x}{http://coq.inria.fr/distrib/8.4pl3/stdlib/Coq.Init.Datatypes}{\coqdocnotation{(}}\coqdocvariable{A}\coqexternalref{:type scope:x '->' x}{http://coq.inria.fr/distrib/8.4pl3/stdlib/Coq.Init.Logic}{\coqdocnotation{\ensuremath{\rightarrow}}}\coqdocvariable{X}\coqexternalref{:type scope:x '*' x}{http://coq.inria.fr/distrib/8.4pl3/stdlib/Coq.Init.Datatypes}{\coqdocnotation{)}} \coqexternalref{:type scope:x '*' x}{http://coq.inria.fr/distrib/8.4pl3/stdlib/Coq.Init.Datatypes}{\coqdocnotation{\ensuremath{\times}}} \coqexternalref{:type scope:x '*' x}{http://coq.inria.fr/distrib/8.4pl3/stdlib/Coq.Init.Datatypes}{\coqdocnotation{(}}\coqdocvariable{B}\coqexternalref{:type scope:x '->' x}{http://coq.inria.fr/distrib/8.4pl3/stdlib/Coq.Init.Logic}{\coqdocnotation{\ensuremath{\rightarrow}}}\coqdocvariable{X}\coqexternalref{:type scope:x '*' x}{http://coq.inria.fr/distrib/8.4pl3/stdlib/Coq.Init.Datatypes}{\coqdocnotation{)}} :=\coqdoceol
\coqdocindent{1.00em}
\coqexternalref{:core scope:'(' x ',' x ',' '..' ',' x ')'}{http://coq.inria.fr/distrib/8.4pl3/stdlib/Coq.Init.Datatypes}{\coqdocnotation{(}}\coqdocvariable{h} \coqdocnotation{o} \coqexternalref{inl}{http://coq.inria.fr/distrib/8.4pl3/stdlib/Coq.Init.Datatypes}{\coqdocconstructor{inl}}\coqexternalref{:core scope:'(' x ',' x ',' '..' ',' x ')'}{http://coq.inria.fr/distrib/8.4pl3/stdlib/Coq.Init.Datatypes}{\coqdocnotation{,}} \coqdocvariable{h} \coqdocnotation{o} \coqexternalref{inr}{http://coq.inria.fr/distrib/8.4pl3/stdlib/Coq.Init.Datatypes}{\coqdocconstructor{inr}}\coqexternalref{:core scope:'(' x ',' x ',' '..' ',' x ')'}{http://coq.inria.fr/distrib/8.4pl3/stdlib/Coq.Init.Datatypes}{\coqdocnotation{)}}.\coqdoceol
\coqdocemptyline
\coqdocnoindent
\coqdockw{Definition} \coqdef{chap02.backward}{backward}{\coqdocdefinition{backward}} \{\coqdocvar{A} \coqdocvar{B} \coqdocvar{X} : \coqdockw{Type}\} (\coqdocvar{h} : \coqexternalref{:type scope:x '*' x}{http://coq.inria.fr/distrib/8.4pl3/stdlib/Coq.Init.Datatypes}{\coqdocnotation{(}}\coqdocvariable{A}\coqexternalref{:type scope:x '->' x}{http://coq.inria.fr/distrib/8.4pl3/stdlib/Coq.Init.Logic}{\coqdocnotation{\ensuremath{\rightarrow}}}\coqdocvariable{X}\coqexternalref{:type scope:x '*' x}{http://coq.inria.fr/distrib/8.4pl3/stdlib/Coq.Init.Datatypes}{\coqdocnotation{)}} \coqexternalref{:type scope:x '*' x}{http://coq.inria.fr/distrib/8.4pl3/stdlib/Coq.Init.Datatypes}{\coqdocnotation{\ensuremath{\times}}} \coqexternalref{:type scope:x '*' x}{http://coq.inria.fr/distrib/8.4pl3/stdlib/Coq.Init.Datatypes}{\coqdocnotation{(}}\coqdocvariable{B}\coqexternalref{:type scope:x '->' x}{http://coq.inria.fr/distrib/8.4pl3/stdlib/Coq.Init.Logic}{\coqdocnotation{\ensuremath{\rightarrow}}}\coqdocvariable{X}\coqexternalref{:type scope:x '*' x}{http://coq.inria.fr/distrib/8.4pl3/stdlib/Coq.Init.Datatypes}{\coqdocnotation{)}}) : \coqdocvariable{A}\coqexternalref{:type scope:x '+' x}{http://coq.inria.fr/distrib/8.4pl3/stdlib/Coq.Init.Datatypes}{\coqdocnotation{+}}\coqdocvariable{B} \coqexternalref{:type scope:x '->' x}{http://coq.inria.fr/distrib/8.4pl3/stdlib/Coq.Init.Logic}{\coqdocnotation{\ensuremath{\rightarrow}}} \coqdocvariable{X} :=\coqdoceol
\coqdocindent{1.00em}
\coqdockw{fun} \coqdocvar{x} \ensuremath{\Rightarrow} \coqdockw{match} \coqdocvariable{x} \coqdockw{with}\coqdoceol
\coqdocindent{6.50em}
\ensuremath{|} \coqexternalref{inl}{http://coq.inria.fr/distrib/8.4pl3/stdlib/Coq.Init.Datatypes}{\coqdocconstructor{inl}} \coqdocvar{a} \ensuremath{\Rightarrow} (\coqexternalref{fst}{http://coq.inria.fr/distrib/8.4pl3/stdlib/Coq.Init.Datatypes}{\coqdocdefinition{fst}} \coqdocvariable{h}) \coqdocvar{a}\coqdoceol
\coqdocindent{6.50em}
\ensuremath{|} \coqexternalref{inr}{http://coq.inria.fr/distrib/8.4pl3/stdlib/Coq.Init.Datatypes}{\coqdocconstructor{inr}} \coqdocvar{b} \ensuremath{\Rightarrow} (\coqexternalref{snd}{http://coq.inria.fr/distrib/8.4pl3/stdlib/Coq.Init.Datatypes}{\coqdocdefinition{snd}} \coqdocvariable{h}) \coqdocvar{b}\coqdoceol
\coqdocindent{5.50em}
\coqdockw{end}.\coqdoceol
\coqdocemptyline
\coqdocnoindent
\coqdockw{Lemma} \coqdef{chap02.alpha2 9}{alpha2\_9}{\coqdoclemma{alpha2\_9}} \{\coqdocvar{A} \coqdocvar{B} \coqdocvar{X}: \coqdockw{Type}\} : \coqdockw{\ensuremath{\forall}} (\coqdocvar{h} : \coqexternalref{:type scope:x '*' x}{http://coq.inria.fr/distrib/8.4pl3/stdlib/Coq.Init.Datatypes}{\coqdocnotation{(}}\coqdocvariable{A} \coqexternalref{:type scope:x '->' x}{http://coq.inria.fr/distrib/8.4pl3/stdlib/Coq.Init.Logic}{\coqdocnotation{\ensuremath{\rightarrow}}} \coqdocvariable{X}\coqexternalref{:type scope:x '*' x}{http://coq.inria.fr/distrib/8.4pl3/stdlib/Coq.Init.Datatypes}{\coqdocnotation{)}} \coqexternalref{:type scope:x '*' x}{http://coq.inria.fr/distrib/8.4pl3/stdlib/Coq.Init.Datatypes}{\coqdocnotation{\ensuremath{\times}}} \coqexternalref{:type scope:x '*' x}{http://coq.inria.fr/distrib/8.4pl3/stdlib/Coq.Init.Datatypes}{\coqdocnotation{(}}\coqdocvariable{B} \coqexternalref{:type scope:x '->' x}{http://coq.inria.fr/distrib/8.4pl3/stdlib/Coq.Init.Logic}{\coqdocnotation{\ensuremath{\rightarrow}}} \coqdocvariable{X}\coqexternalref{:type scope:x '*' x}{http://coq.inria.fr/distrib/8.4pl3/stdlib/Coq.Init.Datatypes}{\coqdocnotation{)}}), \coqdoceol
\coqdocindent{1.00em}
\coqref{chap02.forward}{\coqdocdefinition{forward}} (\coqref{chap02.backward}{\coqdocdefinition{backward}} \coqdocvariable{h}) \coqdocnotation{=} \coqdocvariable{h}.\coqdoceol
\coqdocnoindent
\coqdockw{Proof}.\coqdoceol
\coqdocindent{1.00em}
\coqdoctac{unfold} \coqref{chap02.forward}{\coqdocdefinition{forward}}, \coqref{chap02.backward}{\coqdocdefinition{backward}}. \coqdoctac{destruct} \coqdocvar{h} \coqdockw{as} (\coqdocvar{x}, \coqdocvar{y}). \coqdoctac{reflexivity}.\coqdoceol
\coqdocnoindent
\coqdockw{Qed}.\coqdoceol
\coqdocemptyline
\coqdocnoindent
\coqdockw{Lemma} \coqdef{chap02.beta2 9}{beta2\_9}{\coqdoclemma{beta2\_9}} `\{\coqdocclass{Funext}\} \{\coqdocvar{A} \coqdocvar{B} \coqdocvar{X}: \coqdockw{Type}\} : \coqdockw{\ensuremath{\forall}} (\coqdocvar{h} : \coqdocvariable{A}\coqexternalref{:type scope:x '+' x}{http://coq.inria.fr/distrib/8.4pl3/stdlib/Coq.Init.Datatypes}{\coqdocnotation{+}}\coqdocvariable{B} \coqexternalref{:type scope:x '->' x}{http://coq.inria.fr/distrib/8.4pl3/stdlib/Coq.Init.Logic}{\coqdocnotation{\ensuremath{\rightarrow}}} \coqdocvariable{X}), \coqdoceol
\coqdocindent{1.00em}
\coqref{chap02.backward}{\coqdocdefinition{backward}} (\coqref{chap02.forward}{\coqdocdefinition{forward}} \coqdocvariable{h}) \coqdocnotation{=} \coqdocvariable{h}.\coqdoceol
\coqdocnoindent
\coqdockw{Proof}.\coqdoceol
\coqdocindent{1.00em}
\coqdoctac{intros}. \coqdoctac{apply} \coqdocvar{H}. \coqdoctac{unfold} \coqdocdefinition{pointwise\_paths}. \coqdoctac{intros}. \coqdoctac{destruct} \coqdocvar{x}; \coqdoctac{reflexivity}.\coqdoceol
\coqdocnoindent
\coqdockw{Qed}.\coqdoceol
\coqdocemptyline
\coqdocnoindent
\coqdockw{Theorem} \coqdef{chap02.ex2 9}{ex2\_9}{\coqdoclemma{ex2\_9}} : \coqdockw{\ensuremath{\forall}} \coqdocvar{A} \coqdocvar{B} \coqdocvar{X}, \coqdocclass{IsEquiv}(@\coqref{chap02.forward}{\coqdocdefinition{forward}} \coqdocvariable{A} \coqdocvariable{B} \coqdocvariable{X}).\coqdoceol
\coqdocnoindent
\coqdockw{Proof}.\coqdoceol
\coqdocindent{1.00em}
\coqdoctac{intros}. \coqdoctac{apply} (\coqdocdefinition{isequiv\_adjointify} \coqref{chap02.forward}{\coqdocdefinition{forward}} \coqref{chap02.backward}{\coqdocdefinition{backward}} \coqref{chap02.alpha2 9}{\coqdoclemma{alpha2\_9}} \coqref{chap02.beta2 9}{\coqdoclemma{beta2\_9}}).\coqdoceol
\coqdocnoindent
\coqdockw{Qed}.\coqdoceol
\coqdocemptyline
\coqdocemptyline
\end{coqdoccode}
\exer{2.10}{104} 
Prove that $\Sigma$-types are ``associative'', in that for any $A:\UU$ and
families $B : A \to \UU$ and $C : (\sm{x:A} B(x)) \to \UU$, we have
\[
  \left(\sm{x:A}\sm{y:B(x)}C((x, y))\right)
  \eqvsym
  \left(\sm{p:\sm{x:A}B(x)}C(p)\right)
\]


 \exer{2.11}{104} 
A (homotopy) commutative square
\[
  \xymatrix{
  P \ar[r]^{h} \ar[d]_{k} & A \ar[d]^{f} \\
  B \ar[r]_{g} & C
  }
\]
consists of functions $f$, $g$, $h$, and $k$ as shown, together with a path $f
\circ h = g \circ k$.  Note that this is exactly an element of the pullback $(P
\to A) \times_{P \to C} (P \to B)$ as defined in 2.15.11.  A commutative square
is called a (homotopy) pullback square if for any $X$, the induced map
\[
  (X \to P) \to (X \to A) \times_{(X \to C)} (X \to B)
\]
is an equivalence.  Prove that the pullback $P \defeq A \times_{C} B$ defined
in 2.15.11 is the corner of a pullback square.


 \soln
To show that $P$ is the corner of a pullback square, we need to produce 


 \exer{2.12}{104} 
Suppose given two commutative squares
\[\xymatrix{
  A \ar[r] \ar[d] & C \ar[r] \ar[d] & E \ar[d] \\
  B \ar[r] & D \ar[r] & F
}\]
and suppose that the right-hand square is a pullback square.  Prove that the
left-hand square is a pullback square if and only if the outer rectangle is a
pullback square.


 \soln
The good ol' pullback lemma.


 \exer{2.13}{104} 
Show that $(\bool \eqvsym \bool) \eqvsym \bool$.


 \exer{2.14}{104} 
Suppose we add to type theory the equality reflection rule which says that if
there is an element $p : x = y$, then in fact $x \equiv y$.  Prove that for any
$p : x = x$ we have $p \equiv \refl{x}$.


 \exer{2.15}{105}
Show that Lemma 2.10.5 can be strengthened to
\[
  \transfib{B}{p}{-} =_{B(x)\to B(y)} \idtoeqv(\mapfunc{B}(p))
\]
without using function extensionality.


 \exer{2.16}{105}
Suppose that rather than function extensionality, we suppose only the existence
of an element
\[
  \funext : \prd{A:\UU}\prd{B:A\to\UU}\prd{f, g : \prd{x:A}B(x)}
    (f \sim g) \to (f = g)
\]
(with no relationship to $\happly$ assumed).  Prove that in fact, this is
sufficient to imply the whole function extensionality axiom (that $\happly$ is
an equivalence).


 \exer{2.17}{105}
\begin{enumerate}
  \item Show that if $A \eqvsym A'$ and $B \eqvsym B'$, then $(A \times B)
      \eqvsym (A' \times B')$.
  \item Give two proofs of this fact, one using univalence and one not using
  it, and show that the two proofs are equal.
  \item Formulate and prove analogous results for the other type formers:
  $\Sigma$, $\to$, $\Pi$, and $+$.
\end{enumerate}
\begin{coqdoccode}
\end{coqdoccode}
