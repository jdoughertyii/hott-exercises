\begin{coqdoccode}
\end{coqdoccode}
\section{Category theory}

\begin{coqdoccode}
\coqdocemptyline
\coqdocnoindent
\coqdockw{Require} \coqdockw{Import} \coqdoclibrary{Category}.\coqdoceol
\coqdocnoindent
\coqdockw{Local}\coqdocindent{0.50em}
\coqdockw{Open} \coqdockw{Scope} \coqdocvar{morphism\_scope}.\coqdoceol
\coqdocnoindent
\coqdockw{Local}\coqdocindent{0.50em}
\coqdockw{Open} \coqdockw{Scope} \coqdocvar{category\_scope}.\coqdoceol
\coqdocemptyline
\end{coqdoccode}
\exer{9.1}{334} 
For a precategory $A$ and $a : A$, define the slice precategory $A / a$.  Show
that if $A$ is a category, so is $A / a$.


 \soln
For the type of objects $(A / a)_{0}$, take 
\[
  (A / a)_{0} \defeq \sm{b : A} \hom_{A}(b, a)
\]
For $(b, f), (c, g) : (A / a)$, a morphism $h$ is given by the commutative
triangle
\[\xymatrix{
  b \ar[dr]_{f} \ar[rr]^{h} & & c \ar[dl]^{g} \\
  & a &
}\]
so
\[
  \hom_{(A/a)}((b, f), (c, g)) \defeq \sm{h : \hom_{A}(b, c)} (f = g \circ h)
\]
The identity morphisms $1_{(b,f)} : \hom_{A/a}((b, f), (b, f))$ are given by
$1_{b} : \hom_{A}(b, b)$, along with the proof that $f = f \circ 1_{b}$ from
the precategory $A$.  Composition in $A / a$ is just composition in $A$,
along with associativity of composition in $A$.  Since the unit and composition
in $A / a$ are really just those from $A$ with some contractible data added,
axioms (v) and (vi) follow directly.


The interesting bit is to show that if $A$ is a category, then so is $A / a$.
So suppose that there is an equivalence
\[
  \idtoiso_{A} : \eqv{(a =_{A} b)}{(a \cong_{A} b)}
\]
We want to construct a map
\[
  \left((b, f) \cong_{A/a} (c, g)\right) \to \left((b, f) =_{A/a} (c, g)\right)
\]
that is a quasi-inverse to $\idtoiso_{A/a}$.  So suppose that $i : ((b, f)
\cong (c, g))$.  Then $\fst(i) : \hom_{A}(b, c)$ is also iso, with inverse
$\fst(i^{-1}) : \hom_{A}(c, b)$.  Since $\idtoiso_{A}$ is an equivalence, we
obtain an element $\isotoid_{A}(\fst(i)) : b =_{A} c$.  So now we need to
show that
\[
  \transfib{\hom_{A}(-, a)}{\isotoid_{A}(\fst(i))}{f} = g = f \circ
  \fst(i^{-1})
\]
by the fact that $i$ is iso.  By Lemma 9.1.9, we can rewrite the left as
\[
f \circ \idtoiso_{A}(\isotoid_{A}(\fst(i)))^{-1} = f \circ
  \fst(i^{-1})
\]
and this follows from $\fst(i)^{-1} = \fst(i^{-1})$ and $\idtoiso \circ
\isotoid \sim \idfunc{}$.


To show that this is a quasi-inverse, suppose that $i : ((b, f) \cong (c, g))$,
and let $F(i) : ((b, f) = (c, g))$ be the map we just constructed.  We want to
show that $\idtoiso_{A/a}(F(i)) = i$.  An isomorphism is determined by its
underlying map, and an arrow in $A/a$ is determined by the underlying arrow in
$A$, so we just need to show that $\fst(\idtoiso_{A/a}(F(i))) = \fst(i)$.
\begin{coqdoccode}
\coqdocemptyline
\coqdocnoindent
\coqdockw{Module} \coqdef{Ch09.my slice}{my\_slice}{\coqdocmodule{my\_slice}}.\coqdoceol
\coqdocemptyline
\coqdocnoindent
\coqdockw{Section} \coqdef{Ch09.my slice.my slice parts}{my\_slice\_parts}{\coqdocsection{my\_slice\_parts}}.\coqdoceol
\coqdocnoindent
\coqdockw{Variable} \coqdef{Ch09.my slice.my slice parts.A}{A}{\coqdocvariable{A}} : \coqdocrecord{PreCategory}.\coqdoceol
\coqdocnoindent
\coqdockw{Variable} \coqdef{Ch09.my slice.my slice parts.a}{a}{\coqdocvariable{a}} : \coqdocvariable{A}.\coqdoceol
\coqdocemptyline
\coqdocnoindent
\coqdockw{Record} \coqdef{Ch09.my slice.object}{object}{\coqdocrecord{object}} := \coqdoceol
\coqdocindent{1.00em}
\{\coqdef{Ch09.my slice.b}{b}{\coqdocprojection{b}} : \coqdocvariable{A};\coqdoceol
\coqdocindent{1.50em}
\coqdef{Ch09.my slice.f}{f}{\coqdocprojection{f}} : \coqdocprojection{morphism} \coqdocvariable{A} \coqref{Ch09.b}{\coqdocmethod{b}} \coqdocvariable{a}\}.\coqdoceol
\coqdocemptyline
\coqdocnoindent
\coqdockw{Local}\coqdocindent{0.50em}
\coqdockw{Notation} \coqdef{Ch09.my slice.object sig T}{object\_sig\_T}{\coqdocabbreviation{object\_sig\_T}} := (\coqexternalref{:type scope:'x7B' x ':' x '|' x 'x7D'}{http://coq.inria.fr/distrib/8.4pl4/stdlib/Coq.Init.Specif}{\coqdocnotation{\{}}\coqdocvar{b} \coqexternalref{:type scope:'x7B' x ':' x '|' x 'x7D'}{http://coq.inria.fr/distrib/8.4pl4/stdlib/Coq.Init.Specif}{\coqdocnotation{:}} \coqdocvariable{A} \coqexternalref{:type scope:'x7B' x ':' x '|' x 'x7D'}{http://coq.inria.fr/distrib/8.4pl4/stdlib/Coq.Init.Specif}{\coqdocnotation{\ensuremath{|}}} \coqdocprojection{morphism} \coqdocvariable{A} \coqdocvar{b} \coqdocvariable{a}\coqexternalref{:type scope:'x7B' x ':' x '|' x 'x7D'}{http://coq.inria.fr/distrib/8.4pl4/stdlib/Coq.Init.Specif}{\coqdocnotation{\}}}).\coqdoceol
\coqdocemptyline
\coqdocnoindent
\coqdockw{Lemma} \coqdef{Ch09.my slice.issig object}{issig\_object}{\coqdoclemma{issig\_object}} : \coqref{Ch09.my slice.object sig T}{\coqdocabbreviation{object\_sig\_T}} \coqdocnotation{\ensuremath{\eqvsym}} \coqref{Ch09.my slice.object}{\coqdocrecord{object}}.\coqdoceol
\coqdocnoindent
\coqdockw{Proof}.\coqdoceol
\coqdocindent{1.00em}
\coqdocvar{issig} (@\coqref{Ch09.my slice.Build object}{\coqdocconstructor{Build\_object}}) (@\coqref{Ch09.my slice.b}{\coqdocprojection{b}}) (@\coqref{Ch09.my slice.f}{\coqdocprojection{f}}).\coqdoceol
\coqdocnoindent
\coqdockw{Defined}.\coqdoceol
\coqdocemptyline
\coqdocnoindent
\coqdockw{Lemma} \coqdef{Ch09.my slice.path object}{path\_object}{\coqdoclemma{path\_object}} (\coqdocvar{bf} \coqdocvar{cg} : \coqref{Ch09.my slice.object}{\coqdocrecord{object}})\coqdoceol
\coqdocnoindent
: \coqdocnotation{\ensuremath{\forall}} \coqdocnotation{(}\coqdocvar{eq} : \coqdocvariable{bf}.(\coqref{Ch09.my slice.b}{\coqdocprojection{b}}) \coqdocnotation{=} \coqdocvariable{cg}.(\coqref{Ch09.my slice.b}{\coqdocprojection{b}})\coqdocnotation{),} \coqdoceol
\coqdocindent{2.00em}
\coqdocdefinition{transport} (\coqdockw{fun} \coqdocvar{X} \ensuremath{\Rightarrow} \coqdocprojection{morphism} \coqdocvariable{A} \coqdocvariable{X} \coqdocvar{\_}) \coqdocvariable{eq} \coqdocvariable{bf}.(\coqref{Ch09.my slice.f}{\coqdocprojection{f}}) \coqdocnotation{=} \coqdocvariable{cg}.(\coqref{Ch09.my slice.f}{\coqdocprojection{f}})\coqdoceol
\coqdocindent{2.00em}
\coqexternalref{:type scope:x '->' x}{http://coq.inria.fr/distrib/8.4pl4/stdlib/Coq.Init.Logic}{\coqdocnotation{\ensuremath{\rightarrow}}} \coqdocvariable{bf} \coqdocnotation{=} \coqdocvariable{cg}.\coqdoceol
\coqdocnoindent
\coqdockw{Proof}.\coqdoceol
\coqdocindent{1.00em}
\coqdoctac{destruct} \coqdocvar{bf}, \coqdocvar{cg}; \coqdoctac{simpl}.\coqdoceol
\coqdocindent{1.00em}
\coqdoctac{intros}; \coqdocvar{path\_induction}; \coqdoctac{reflexivity}.\coqdoceol
\coqdocnoindent
\coqdockw{Defined}.\coqdoceol
\coqdocemptyline
\coqdocnoindent
\coqdockw{Definition} \coqdef{Ch09.my slice.path object uncurried}{path\_object\_uncurried}{\coqdocdefinition{path\_object\_uncurried}} (\coqdocvar{bf} \coqdocvar{cg} : \coqref{Ch09.my slice.object}{\coqdocrecord{object}})\coqdoceol
\coqdocindent{1.00em}
: \coqexternalref{:type scope:'x7B' x ':' x 'x26' x 'x7D'}{http://coq.inria.fr/distrib/8.4pl4/stdlib/Coq.Init.Specif}{\coqdocnotation{\{}}\coqdocvar{eq} \coqexternalref{:type scope:'x7B' x ':' x 'x26' x 'x7D'}{http://coq.inria.fr/distrib/8.4pl4/stdlib/Coq.Init.Specif}{\coqdocnotation{:}} \coqdocvariable{bf}.(\coqref{Ch09.my slice.b}{\coqdocprojection{b}}) \coqdocnotation{=} \coqdocvariable{cg}.(\coqref{Ch09.my slice.b}{\coqdocprojection{b}}) \coqexternalref{:type scope:'x7B' x ':' x 'x26' x 'x7D'}{http://coq.inria.fr/distrib/8.4pl4/stdlib/Coq.Init.Specif}{\coqdocnotation{\&}}\coqdoceol
\coqdocindent{3.00em}
\coqdocdefinition{transport} (\coqdockw{fun} \coqdocvar{X} \ensuremath{\Rightarrow} \coqdocprojection{morphism} \coqdocvariable{A} \coqdocvariable{X} \coqdocvar{\_}) \coqdocvar{eq} \coqdocvariable{bf}.(\coqref{Ch09.my slice.f}{\coqdocprojection{f}}) \coqdocnotation{=} \coqdocvariable{cg}.(\coqref{Ch09.my slice.f}{\coqdocprojection{f}})\coqexternalref{:type scope:'x7B' x ':' x 'x26' x 'x7D'}{http://coq.inria.fr/distrib/8.4pl4/stdlib/Coq.Init.Specif}{\coqdocnotation{\}}} \coqdoceol
\coqdocindent{2.00em}
\coqexternalref{:type scope:x '->' x}{http://coq.inria.fr/distrib/8.4pl4/stdlib/Coq.Init.Logic}{\coqdocnotation{\ensuremath{\rightarrow}}} \coqdocvariable{bf} \coqdocnotation{=} \coqdocvariable{cg}\coqdoceol
\coqdocindent{1.00em}
:= \coqdockw{fun} \coqdocvar{H} \ensuremath{\Rightarrow} \coqref{Ch09.my slice.path object}{\coqdoclemma{path\_object}} \coqdocvariable{bf} \coqdocvariable{cg} \coqdocvariable{H}\coqdocnotation{.1} \coqdocvariable{H}\coqdocnotation{.2}.\coqdoceol
\coqdocemptyline
\coqdocnoindent
\coqdockw{Record} \coqdef{Ch09.my slice.morphism}{morphism}{\coqdocrecord{morphism}} (\coqdocvar{bf} \coqdocvar{cg} : \coqref{Ch09.my slice.object}{\coqdocrecord{object}}) := \coqdoceol
\coqdocindent{1.00em}
\{\coqdef{Ch09.my slice.h}{h}{\coqdocprojection{h}} : \coqdocprojection{Category.Core.morphism} \coqdocvariable{A} (\coqdocvariable{bf}.(\coqref{Ch09.my slice.b}{\coqdocprojection{b}})) (\coqdocvariable{cg}.(\coqref{Ch09.my slice.b}{\coqdocprojection{b}}));\coqdoceol
\coqdocindent{1.50em}
\coqdef{Ch09.my slice.p}{p}{\coqdocprojection{p}} : \coqdocnotation{(}\coqdocvariable{bf}.(\coqref{Ch09.my slice.f}{\coqdocprojection{f}})\coqdocnotation{)} \coqdocnotation{=} \coqdocnotation{(}\coqdocvariable{cg}.(\coqref{Ch09.my slice.f}{\coqdocprojection{f}})\coqdocnotation{)} \coqdocnotation{o} \coqref{Ch09.h}{\coqdocmethod{h}}\}.\coqdoceol
\coqdocemptyline
\coqdocnoindent
\coqdockw{Local}\coqdocindent{0.50em}
\coqdockw{Notation} \coqdef{Ch09.my slice.morphism sig T}{morphism\_sig\_T}{\coqdocabbreviation{morphism\_sig\_T}} \coqdocvar{bf} \coqdocvar{cg} :=\coqdoceol
\coqdocindent{1.00em}
(\coqexternalref{:type scope:'x7B' x ':' x '|' x 'x7D'}{http://coq.inria.fr/distrib/8.4pl4/stdlib/Coq.Init.Specif}{\coqdocnotation{\{}} \coqdocvar{h} \coqexternalref{:type scope:'x7B' x ':' x '|' x 'x7D'}{http://coq.inria.fr/distrib/8.4pl4/stdlib/Coq.Init.Specif}{\coqdocnotation{:}} \coqdocprojection{Category.Core.morphism} \coqdocvariable{A} (\coqdocvar{bf}.(\coqref{Ch09.my slice.b}{\coqdocprojection{b}})) (\coqdocvar{cg}.(\coqref{Ch09.my slice.b}{\coqdocprojection{b}}))\coqdoceol
\coqdocindent{1.50em}
\coqexternalref{:type scope:'x7B' x ':' x '|' x 'x7D'}{http://coq.inria.fr/distrib/8.4pl4/stdlib/Coq.Init.Specif}{\coqdocnotation{\ensuremath{|}}} \coqdocnotation{(}\coqdocvar{bf}.(\coqref{Ch09.my slice.f}{\coqdocprojection{f}})\coqdocnotation{)} \coqdocnotation{=} \coqdocnotation{(}\coqdocvar{cg}.(\coqref{Ch09.my slice.f}{\coqdocprojection{f}})\coqdocnotation{)} \coqdocnotation{o} \coqdocvar{h}\coqexternalref{:type scope:'x7B' x ':' x '|' x 'x7D'}{http://coq.inria.fr/distrib/8.4pl4/stdlib/Coq.Init.Specif}{\coqdocnotation{\}}}).\coqdoceol
\coqdocemptyline
\coqdocnoindent
\coqdockw{Lemma} \coqdef{Ch09.my slice.issig morphism}{issig\_morphism}{\coqdoclemma{issig\_morphism}} \coqdocvar{bf} \coqdocvar{cg} : \coqdocnotation{(}\coqref{Ch09.my slice.morphism sig T}{\coqdocabbreviation{morphism\_sig\_T}} \coqdocvariable{bf} \coqdocvariable{cg}\coqdocnotation{)} \coqdocnotation{\ensuremath{\eqvsym}} \coqref{Ch09.my slice.morphism}{\coqdocrecord{morphism}} \coqdocvariable{bf} \coqdocvariable{cg}.\coqdoceol
\coqdocnoindent
\coqdockw{Proof}.\coqdoceol
\coqdocindent{1.00em}
\coqdocvar{issig} (@\coqref{Ch09.my slice.Build morphism}{\coqdocconstructor{Build\_morphism}} \coqdocvar{bf} \coqdocvar{cg})\coqdoceol
\coqdocindent{4.00em}
(\coqref{Ch09.my slice.h}{\coqdocprojection{h}} \coqdocvar{bf} \coqdocvar{cg})\coqdoceol
\coqdocindent{4.00em}
(\coqref{Ch09.my slice.p}{\coqdocprojection{p}} \coqdocvar{bf} \coqdocvar{cg}).\coqdoceol
\coqdocnoindent
\coqdockw{Defined}.\coqdoceol
\coqdocemptyline
\coqdocnoindent
\coqdockw{Lemma} \coqdef{Ch09.my slice.path morphism}{path\_morphism}{\coqdoclemma{path\_morphism}} \coqdocvar{bf} \coqdocvar{cg} (\coqdocvar{ip} \coqdocvar{jq} : \coqref{Ch09.my slice.morphism}{\coqdocrecord{morphism}} \coqdocvariable{bf} \coqdocvariable{cg})\coqdoceol
\coqdocindent{1.00em}
: \coqdocnotation{(}\coqref{Ch09.my slice.h}{\coqdocprojection{h}} \coqdocvar{\_} \coqdocvar{\_} \coqdocvariable{ip}\coqdocnotation{)} \coqdocnotation{=} \coqdocnotation{(}\coqref{Ch09.my slice.h}{\coqdocprojection{h}} \coqdocvar{\_} \coqdocvar{\_} \coqdocvariable{jq}\coqdocnotation{)} \coqexternalref{:type scope:x '->' x}{http://coq.inria.fr/distrib/8.4pl4/stdlib/Coq.Init.Logic}{\coqdocnotation{\ensuremath{\rightarrow}}} \coqdocvariable{ip} \coqdocnotation{=} \coqdocvariable{jq}.\coqdoceol
\coqdocnoindent
\coqdockw{Proof}.\coqdoceol
\coqdocindent{1.00em}
\coqdoctac{destruct} \coqdocvar{ip}, \coqdocvar{jq}; \coqdoctac{simpl}.\coqdoceol
\coqdocindent{1.00em}
\coqdoctac{intro}; \coqdocvar{path\_induction}.\coqdoceol
\coqdocindent{1.00em}
\coqdocvar{f\_ap}.\coqdoceol
\coqdocindent{1.00em}
\coqdoctac{exact} (\coqdocmethod{center} \coqdocvar{\_}).\coqdoceol
\coqdocnoindent
\coqdockw{Defined}.\coqdoceol
\coqdocemptyline
\coqdocnoindent
\coqdockw{Definition} \coqdef{Ch09.my slice.compose}{compose}{\coqdocdefinition{compose}} \coqdocvar{bf} \coqdocvar{cg} \coqdocvar{dh}\coqdoceol
\coqdocindent{5.50em}
(\coqdocvar{ip} : \coqref{Ch09.my slice.morphism}{\coqdocrecord{morphism}} \coqdocvariable{cg} \coqdocvariable{dh}) (\coqdocvar{jq} : \coqref{Ch09.my slice.morphism}{\coqdocrecord{morphism}} \coqdocvariable{bf} \coqdocvariable{cg})\coqdoceol
\coqdocindent{1.00em}
: \coqref{Ch09.my slice.morphism}{\coqdocrecord{morphism}} \coqdocvariable{bf} \coqdocvariable{dh}.\coqdoceol
\coqdocnoindent
\coqdockw{Proof}.\coqdoceol
\coqdocindent{1.00em}
\coqdoctac{\ensuremath{\exists}} (\coqdocnotation{(}\coqref{Ch09.my slice.h}{\coqdocprojection{h}} \coqdocvar{\_} \coqdocvar{\_} \coqdocvar{ip}\coqdocnotation{)} \coqdocnotation{o} \coqdocnotation{(}\coqref{Ch09.my slice.h}{\coqdocprojection{h}} \coqdocvar{\_} \coqdocvar{\_} \coqdocvar{jq}\coqdocnotation{)}).\coqdoceol
\coqdocindent{1.00em}
\coqdoctac{refine} (\coqdocvar{\_} \coqdocnotation{@} \coqdocnotation{(}\coqdocprojection{associativity} \coqdocvar{\_} \coqdocvar{\_} \coqdocvar{\_} \coqdocvar{\_} \coqdocvar{\_} \coqdocvar{\_} \coqdocvar{\_} \coqdocvar{\_}\coqdocnotation{)}).\coqdoceol
\coqdocindent{1.00em}
\coqdoctac{transitivity} (\coqref{Ch09.my slice.f}{\coqdocprojection{f}} \coqdocvar{cg} \coqdocnotation{o} \coqdocnotation{(}\coqref{Ch09.my slice.h}{\coqdocprojection{h}} \coqdocvar{\_} \coqdocvar{\_} \coqdocvar{jq}\coqdocnotation{)}).\coqdoceol
\coqdocindent{1.00em}
\coqdoctac{apply} \coqref{Ch09.my slice.p}{\coqdocprojection{p}}. \coqdocvar{f\_ap}. \coqdoctac{apply} \coqref{Ch09.my slice.p}{\coqdocprojection{p}}.\coqdoceol
\coqdocnoindent
\coqdockw{Defined}.\coqdoceol
\coqdocemptyline
\coqdocnoindent
\coqdockw{Definition} \coqdef{Ch09.my slice.identity}{identity}{\coqdocdefinition{identity}} \coqdocvar{x} : \coqref{Ch09.my slice.morphism}{\coqdocrecord{morphism}} \coqdocvariable{x} \coqdocvariable{x}.\coqdoceol
\coqdocnoindent
\coqdockw{Proof}.\coqdoceol
\coqdocindent{1.00em}
\coqdoctac{\ensuremath{\exists}} (\coqdocprojection{identity} (\coqdocvar{x}.(\coqref{Ch09.my slice.b}{\coqdocprojection{b}}))).\coqdoceol
\coqdocindent{1.00em}
\coqdoctac{apply} \coqdocnotation{(}\coqdocprojection{right\_identity} \coqdocvar{\_} \coqdocvar{\_} \coqdocvar{\_} \coqdocvar{\_}\coqdocnotation{)\^{}}.\coqdoceol
\coqdocnoindent
\coqdockw{Defined}.\coqdoceol
\coqdocemptyline
\coqdocnoindent
\coqdockw{End} \coqref{Ch09.my slice.my slice parts}{\coqdocsection{my\_slice\_parts}}.\coqdoceol
\coqdocemptyline
\coqdocnoindent
\coqdockw{Local}\coqdocindent{0.50em}
\coqdockw{Ltac} \coqdocvar{path\_slice\_t} :=\coqdoceol
\coqdocindent{1.00em}
\coqdoctac{intros};\coqdoceol
\coqdocindent{1.00em}
\coqdoctac{apply} \coqref{Ch09.my slice.path morphism}{\coqdoclemma{path\_morphism}};\coqdoceol
\coqdocindent{1.00em}
\coqdoctac{simpl};\coqdoceol
\coqdocindent{1.00em}
\coqdoctac{auto} \coqdockw{with} \coqdocvar{morphism}.\coqdoceol
\coqdocemptyline
\coqdocnoindent
\coqdockw{Definition} \coqdef{Ch09.my slice.slice precategory}{slice\_precategory}{\coqdocdefinition{slice\_precategory}} (\coqdocvar{A} : \coqdocrecord{PreCategory}) (\coqdocvar{a} : \coqdocvariable{A}) : \coqdocrecord{PreCategory}.\coqdoceol
\coqdocnoindent
\coqdockw{Proof}.\coqdoceol
\coqdocindent{1.00em}
\coqdoctac{refine} (@\coqdocdefinition{Build\_PreCategory} (\coqref{Ch09.my slice.object}{\coqdocrecord{object}} \coqdocvar{A} \coqdocvar{a})\coqdoceol
\coqdocindent{14.50em}
(\coqref{Ch09.my slice.morphism}{\coqdocrecord{morphism}} \coqdocvar{A} \coqdocvar{a})\coqdoceol
\coqdocindent{14.50em}
(\coqref{Ch09.my slice.identity}{\coqdocdefinition{identity}} \coqdocvar{A} \coqdocvar{a})\coqdoceol
\coqdocindent{14.50em}
(\coqref{Ch09.my slice.compose}{\coqdocdefinition{compose}} \coqdocvar{A} \coqdocvar{a})\coqdoceol
\coqdocindent{14.50em}
\coqdocvar{\_} \coqdocvar{\_} \coqdocvar{\_} \coqdocvar{\_});\coqdoceol
\coqdocindent{1.00em}
\coqdoctac{try} \coqdocvar{path\_slice\_t}.\coqdoceol
\coqdocindent{1.00em}
\coqdoctac{intros} \coqdocvar{s} \coqdocvar{d}. \coqdoctac{refine} (\coqdocdefinition{trunc\_equiv} (\coqref{Ch09.my slice.issig morphism}{\coqdoclemma{issig\_morphism}} \coqdocvar{A} \coqdocvar{a} \coqdocvar{s} \coqdocvar{d})).\coqdoceol
\coqdocnoindent
\coqdockw{Defined}.\coqdoceol
\coqdocemptyline
\coqdocnoindent
\coqdockw{Lemma} \coqdef{Ch09.my slice.sliceiso to iso}{sliceiso\_to\_iso}{\coqdoclemma{sliceiso\_to\_iso}} (\coqdocvar{A} : \coqdocrecord{PreCategory}) (\coqdocvar{a} : \coqdocvariable{A}) \coqdoceol
\coqdocindent{3.00em}
(\coqdocvar{bf} \coqdocvar{cg} : \coqref{Ch09.my slice.slice precategory}{\coqdocdefinition{slice\_precategory}} \coqdocvariable{A} \coqdocvariable{a}) :\coqdoceol
\coqdocindent{1.00em}
\coqexternalref{:type scope:x '->' x}{http://coq.inria.fr/distrib/8.4pl4/stdlib/Coq.Init.Logic}{\coqdocnotation{(}}\coqdocvariable{bf} \coqdocnotation{\ensuremath{\cong}} \coqdocvariable{cg}\coqexternalref{:type scope:x '->' x}{http://coq.inria.fr/distrib/8.4pl4/stdlib/Coq.Init.Logic}{\coqdocnotation{)}} \coqexternalref{:type scope:x '->' x}{http://coq.inria.fr/distrib/8.4pl4/stdlib/Coq.Init.Logic}{\coqdocnotation{\ensuremath{\rightarrow}}} \coqexternalref{:type scope:x '->' x}{http://coq.inria.fr/distrib/8.4pl4/stdlib/Coq.Init.Logic}{\coqdocnotation{(}}\coqdocnotation{(}\coqref{Ch09.my slice.b}{\coqdocprojection{b}} \coqdocvar{\_} \coqdocvar{\_} \coqdocvariable{bf}\coqdocnotation{)} \coqdocnotation{\ensuremath{\cong}} \coqdocnotation{(}\coqref{Ch09.my slice.b}{\coqdocprojection{b}} \coqdocvar{\_} \coqdocvar{\_} \coqdocvariable{cg}\coqdocnotation{)}\coqexternalref{:type scope:x '->' x}{http://coq.inria.fr/distrib/8.4pl4/stdlib/Coq.Init.Logic}{\coqdocnotation{)}}.\coqdoceol
\coqdocnoindent
\coqdockw{Proof}.\coqdoceol
\coqdocindent{1.00em}
\coqdoctac{intro} \coqdocvar{H}.\coqdoceol
\coqdocindent{1.00em}
\coqdoctac{destruct} \coqdocvar{bf} \coqdockw{as} [\coqdocvar{b} \coqdocvar{f}], \coqdocvar{cg} \coqdockw{as} [\coqdocvar{c} \coqdocvar{g}].\coqdoceol
\coqdocindent{1.00em}
\coqdoctac{destruct} \coqdocvar{H} \coqdockw{as} [\coqdocvar{i} \coqdocvar{H}].\coqdoceol
\coqdocindent{1.00em}
\coqdoctac{destruct} \coqdocvar{i} \coqdockw{as} [\coqdocvar{i} \coqdocvar{eq}].\coqdoceol
\coqdocindent{1.00em}
\coqdoctac{apply} \coqdoclemma{issig\_isomorphic}.\coqdoceol
\coqdocindent{1.00em}
\coqdoctac{\ensuremath{\exists}} \coqdocvar{i}.\coqdoceol
\coqdocindent{1.00em}
\coqdoctac{destruct} \coqdocvar{H} \coqdockw{as} [\coqdocvar{ii} \coqdocvar{ri} \coqdocvar{li}].\coqdoceol
\coqdocindent{1.00em}
\coqdoctac{destruct} \coqdocvar{ii} \coqdockw{as} [\coqdocvar{ii} \coqdocvar{eq'}].\coqdoceol
\coqdocindent{1.00em}
\coqdoctac{refine} (@\coqdocconstructor{Category.Build\_IsIsomorphism} \coqdocvar{\_} \coqdocvar{\_} \coqdocvar{\_} \coqdocvar{i} \coqdocvar{\_} \coqdocvar{\_} \coqdocvar{\_}).\coqdoceol
\coqdocindent{1.00em}
\coqdoctac{apply} \coqdocvar{ii}.\coqdoceol
\coqdocindent{1.00em}
\coqdoctac{simpl} \coqdoctac{in} *. \coqdoctac{apply} (\coqdocdefinition{ap} (\coqref{Ch09.my slice.h}{\coqdocprojection{h}} \coqdocvar{\_} \coqdocvar{\_} \coqdocvar{\_} \coqdocvar{\_})) \coqdoctac{in} \coqdocvar{ri}. \coqdoctac{apply} \coqdocvar{ri}.\coqdoceol
\coqdocindent{1.00em}
\coqdoctac{simpl} \coqdoctac{in} *. \coqdoctac{apply} (\coqdocdefinition{ap} (\coqref{Ch09.my slice.h}{\coqdocprojection{h}} \coqdocvar{\_} \coqdocvar{\_} \coqdocvar{\_} \coqdocvar{\_})) \coqdoctac{in} \coqdocvar{li}. \coqdoctac{apply} \coqdocvar{li}.\coqdoceol
\coqdocnoindent
\coqdockw{Defined}.\coqdoceol
\coqdocemptyline
\coqdocemptyline
\coqdocnoindent
\begin{coqdoccomment}
\coqdocindent{0.50em}
This\coqdocindent{0.50em}
should\coqdocindent{0.50em}
really\coqdocindent{0.50em}
just\coqdocindent{0.50em}
be\coqdocindent{0.50em}
an\coqdocindent{0.50em}
application\coqdocindent{0.50em}
of\coqdocindent{0.50em}
\coqdocvar{idtoiso\_of\_transport}\coqdocindent{0.50em}
\end{coqdoccomment}
\coqdoceol
\coqdocnoindent
\coqdockw{Lemma} \coqdef{Ch09.my slice.Lemma9 1 9}{Lemma9\_1\_9}{\coqdoclemma{Lemma9\_1\_9}} (\coqdocvar{A} : \coqdocrecord{PreCategory}) (\coqdocvar{a} \coqdocvar{a'} \coqdocvar{b} \coqdocvar{b'} : \coqdocvariable{A}) (\coqdocvar{p} : \coqdocvariable{a} \coqdocnotation{=} \coqdocvariable{a'}) (\coqdocvar{q} : \coqdocvariable{b} \coqdocnotation{=} \coqdocvariable{b'})\coqdoceol
\coqdocindent{3.00em}
(\coqdocvar{f} : \coqdocprojection{Category.morphism} \coqdocvariable{A} \coqdocvariable{a} \coqdocvariable{b}) :\coqdoceol
\coqdocindent{1.00em}
\coqdocdefinition{transport} (\coqdockw{fun} \coqdocvar{z} : \coqdocvariable{A} \coqexternalref{:type scope:x '*' x}{http://coq.inria.fr/distrib/8.4pl4/stdlib/Coq.Init.Datatypes}{\coqdocnotation{\ensuremath{\times}}} \coqdocvariable{A} \ensuremath{\Rightarrow} \coqdocprojection{Category.morphism} \coqdocvariable{A} (\coqexternalref{fst}{http://coq.inria.fr/distrib/8.4pl4/stdlib/Coq.Init.Datatypes}{\coqdocprojection{fst}} \coqdocvariable{z}) (\coqexternalref{snd}{http://coq.inria.fr/distrib/8.4pl4/stdlib/Coq.Init.Datatypes}{\coqdocprojection{snd}} \coqdocvariable{z})) \coqdoceol
\coqdocindent{6.00em}
(\coqdocdefinition{path\_prod} \coqexternalref{:core scope:'(' x ',' x ',' '..' ',' x ')'}{http://coq.inria.fr/distrib/8.4pl4/stdlib/Coq.Init.Datatypes}{\coqdocnotation{(}}\coqdocvariable{a}\coqexternalref{:core scope:'(' x ',' x ',' '..' ',' x ')'}{http://coq.inria.fr/distrib/8.4pl4/stdlib/Coq.Init.Datatypes}{\coqdocnotation{,}} \coqdocvariable{b}\coqexternalref{:core scope:'(' x ',' x ',' '..' ',' x ')'}{http://coq.inria.fr/distrib/8.4pl4/stdlib/Coq.Init.Datatypes}{\coqdocnotation{)}} \coqexternalref{:core scope:'(' x ',' x ',' '..' ',' x ')'}{http://coq.inria.fr/distrib/8.4pl4/stdlib/Coq.Init.Datatypes}{\coqdocnotation{(}}\coqdocvariable{a'}\coqexternalref{:core scope:'(' x ',' x ',' '..' ',' x ')'}{http://coq.inria.fr/distrib/8.4pl4/stdlib/Coq.Init.Datatypes}{\coqdocnotation{,}} \coqdocvariable{b'}\coqexternalref{:core scope:'(' x ',' x ',' '..' ',' x ')'}{http://coq.inria.fr/distrib/8.4pl4/stdlib/Coq.Init.Datatypes}{\coqdocnotation{)}} \coqdocvariable{p} \coqdocvariable{q}) \coqdocvariable{f}\coqdoceol
\coqdocindent{1.00em}
\coqdocnotation{=}\coqdoceol
\coqdocindent{1.00em}
\coqdocnotation{(}@\coqdocprojection{morphism\_isomorphic} \coqdocvar{\_} \coqdocvar{\_} \coqdocvar{\_} (\coqdocdefinition{idtoiso} \coqdocvar{\_} \coqdocvariable{q})\coqdocnotation{)} \coqdocnotation{o} \coqdocvariable{f} \coqdocnotation{o} \coqdocnotation{(}\coqdocdefinition{idtoiso} \coqdocvar{\_} \coqdocvariable{p}\coqdocnotation{)\^{}-1}.\coqdoceol
\coqdocnoindent
\coqdockw{Proof}.\coqdoceol
\coqdocindent{1.00em}
\coqdocvar{path\_induction}.\coqdoceol
\coqdocindent{1.00em}
\coqdoctac{refine} (\coqdocvar{\_} \coqdocnotation{@} \coqdocnotation{(}\coqdocprojection{right\_identity} \coqdocvar{\_} \coqdocvar{\_} \coqdocvar{\_} \coqdocvar{\_}\coqdocnotation{)\^{}}).\coqdoceol
\coqdocindent{1.00em}
\coqdoctac{refine} (\coqdocvar{\_} \coqdocnotation{@} \coqdocnotation{(}\coqdocprojection{left\_identity} \coqdocvar{\_} \coqdocvar{\_} \coqdocvar{\_} \coqdocvar{\_}\coqdocnotation{)\^{}}).\coqdoceol
\coqdocindent{1.00em}
\coqdoctac{reflexivity}.\coqdoceol
\coqdocnoindent
\coqdockw{Defined}.\coqdoceol
\coqdocemptyline
\coqdocnoindent
\begin{coqdoccomment}
\coqdoceol
Theorem\coqdocindent{0.50em}
slice\_cat\_if\_cat\coqdocindent{0.50em}
(A\coqdocindent{0.50em}
:\coqdocindent{0.50em}
PreCategory)\coqdocindent{0.50em}
(a\coqdocindent{0.50em}
:\coqdocindent{0.50em}
A)\coqdocindent{0.50em}
\coqdoceol
\coqdocindent{1.00em}
:\coqdocindent{0.50em}
(IsCategory\coqdocindent{0.50em}
A)\coqdocindent{0.50em}
->\coqdocindent{0.50em}
(IsCategory\coqdocindent{0.50em}
(slice\_precategory\coqdocindent{0.50em}
A\coqdocindent{0.50em}
a)).\coqdoceol
Proof.\coqdoceol
\coqdocindent{1.00em}
intro\coqdocindent{0.50em}
H.\coqdoceol
\coqdocindent{1.00em}
intros\coqdocindent{0.50em}
bf\coqdocindent{0.50em}
cg.\coqdoceol
\coqdocindent{1.00em}
destruct\coqdocindent{0.50em}
bf\coqdocindent{0.50em}
as\coqdocindent{0.50em}
\coqdocvar{b} \coqdocvar{f},\coqdocindent{0.50em}
cg\coqdocindent{0.50em}
as\coqdocindent{0.50em}
\coqdocvar{c} \coqdocvar{g}.\coqdoceol
\coqdocindent{1.00em}
refine\coqdocindent{0.50em}
(isequiv\_adjointify\coqdocindent{0.50em}
\_\coqdocindent{0.50em}
\_\coqdocindent{0.50em}
\_\coqdocindent{0.50em}
\_).\coqdoceol
\coqdocindent{1.00em}
intro\coqdocindent{0.50em}
iso.\coqdocindent{0.50em}
\coqdoceol
\coqdocindent{1.00em}
destruct\coqdocindent{0.50em}
iso\coqdocindent{0.50em}
as\coqdocindent{0.50em}
[\coqdocvar{iso} \coqdocvar{iso\_comm}] [[\coqdocvar{iso\_inv} \coqdocvar{iso\_inv\_comm}] \coqdocvar{r\_inv} \coqdocvar{l\_inv}].\coqdoceol
\coqdocindent{1.00em}
apply\coqdocindent{0.50em}
path\_object\_uncurried.\coqdocindent{0.50em}
\coqdoceol
\coqdocindent{1.00em}
simpl\coqdocindent{0.50em}
in\coqdocindent{0.50em}
*.\coqdoceol
\coqdocindent{1.00em}
transparent\coqdocindent{0.50em}
assert\coqdocindent{0.50em}
(i\coqdocindent{0.50em}
:\coqdocindent{0.50em}
(b\coqdocindent{0.50em}
<\~{}=\~{}>\coqdocindent{0.50em}
c)).\coqdoceol
\coqdocindent{2.00em}
refine\coqdocindent{0.50em}
(@Build\_Isomorphic\coqdocindent{0.50em}
\_\coqdocindent{0.50em}
b\coqdocindent{0.50em}
c\coqdocindent{0.50em}
iso\coqdocindent{0.50em}
\_).\coqdoceol
\coqdocindent{2.00em}
refine\coqdocindent{0.50em}
(Build\_IsIsomorphism\coqdocindent{0.50em}
\_\coqdocindent{0.50em}
\_\coqdocindent{0.50em}
\_\coqdocindent{0.50em}
\_\coqdocindent{0.50em}
iso\_inv\coqdocindent{0.50em}
\_\coqdocindent{0.50em}
\_).\coqdoceol
\coqdocindent{2.00em}
apply\coqdocindent{0.50em}
(ap\coqdocindent{0.50em}
(h\coqdocindent{0.50em}
\_\coqdocindent{0.50em}
\_\coqdocindent{0.50em}
\_\coqdocindent{0.50em}
\_)\coqdocindent{0.50em}
r\_inv).\coqdocindent{0.50em}
apply\coqdocindent{0.50em}
(ap\coqdocindent{0.50em}
(h\coqdocindent{0.50em}
\_\coqdocindent{0.50em}
\_\coqdocindent{0.50em}
\_\coqdocindent{0.50em}
\_)\coqdocindent{0.50em}
l\_inv).\coqdoceol
\coqdocindent{1.00em}
transparent\coqdocindent{0.50em}
assert\coqdocindent{0.50em}
(eq\coqdocindent{0.50em}
:\coqdocindent{0.50em}
(b\coqdocindent{0.50em}
=\coqdocindent{0.50em}
c)).\coqdocindent{0.50em}
apply\coqdocindent{0.50em}
H.\coqdocindent{0.50em}
apply\coqdocindent{0.50em}
i.\coqdocindent{0.50em}
exists\coqdocindent{0.50em}
eq.\coqdocindent{0.50em}
\coqdoceol
\coqdocindent{1.00em}
unfold\coqdocindent{0.50em}
eq.\coqdocindent{0.50em}
\coqdoceol
\coqdocindent{1.00em}
transitivity\coqdocindent{0.50em}
(\coqdoceol
\coqdocindent{2.00em}
transport\coqdocindent{0.50em}
(fun\coqdocindent{0.50em}
z\coqdocindent{0.50em}
=>\coqdocindent{0.50em}
Category.morphism\coqdocindent{0.50em}
A\coqdocindent{0.50em}
(fst\coqdocindent{0.50em}
z)\coqdocindent{0.50em}
(snd\coqdocindent{0.50em}
z))\coqdocindent{0.50em}
\coqdoceol
\coqdocindent{7.00em}
(path\_prod\coqdocindent{0.50em}
(b,\coqdocindent{0.50em}
a)\coqdocindent{0.50em}
(c,\coqdocindent{0.50em}
a)\coqdocindent{0.50em}
((isotoid\coqdocindent{0.50em}
A\coqdocindent{0.50em}
b\coqdocindent{0.50em}
c)\coqdocindent{0.50em}
i)\coqdocindent{0.50em}
1)\coqdocindent{0.50em}
f\coqdoceol
\coqdocindent{1.00em}
).\coqdoceol
\coqdocindent{1.00em}
refine\coqdocindent{0.50em}
(\_\coqdocindent{0.50em}
@\coqdocindent{0.50em}
(transport\_path\_prod\coqdocindent{0.50em}
\_\coqdocindent{0.50em}
\_\coqdocindent{0.50em}
\_\coqdocindent{0.50em}
\_\coqdocindent{0.50em}
\_\coqdocindent{0.50em}
\_\coqdocindent{0.50em}
\_\coqdocindent{0.50em}
\_)\^{}).\coqdocindent{0.50em}
reflexivity.\coqdoceol
\coqdocindent{1.00em}
refine\coqdocindent{0.50em}
((Lemma9\_1\_9\coqdocindent{0.50em}
\_\coqdocindent{0.50em}
\_\coqdocindent{0.50em}
\_\coqdocindent{0.50em}
\_\coqdocindent{0.50em}
\_\coqdocindent{0.50em}
\_\coqdocindent{0.50em}
\_\coqdocindent{0.50em}
\_)\coqdocindent{0.50em}
@\coqdocindent{0.50em}
\_).\coqdocindent{0.50em}
simpl.\coqdoceol
\coqdocindent{1.00em}
refine\coqdocindent{0.50em}
((associativity\coqdocindent{0.50em}
\_\coqdocindent{0.50em}
\_\coqdocindent{0.50em}
\_\coqdocindent{0.50em}
\_\coqdocindent{0.50em}
\_\coqdocindent{0.50em}
\_\coqdocindent{0.50em}
\_\coqdocindent{0.50em}
\_)\coqdocindent{0.50em}
@\coqdocindent{0.50em}
\_).\coqdoceol
\coqdocindent{1.00em}
refine\coqdocindent{0.50em}
((left\_identity\coqdocindent{0.50em}
\_\coqdocindent{0.50em}
\_\coqdocindent{0.50em}
\_\coqdocindent{0.50em}
\_)\coqdocindent{0.50em}
@\coqdocindent{0.50em}
\_).\coqdoceol
\coqdocindent{1.00em}
refine\coqdocindent{0.50em}
(\_\coqdocindent{0.50em}
@\coqdocindent{0.50em}
iso\_inv\_comm\^{}).\coqdocindent{0.50em}
\coqdoceol
\coqdocindent{1.00em}
f\_ap.\coqdoceol
\coqdocindent{1.00em}
apply\coqdocindent{0.50em}
iso\_moveR\_V1.\coqdoceol
\coqdocindent{1.00em}
refine\coqdocindent{0.50em}
((@right\_inverse\coqdocindent{0.50em}
\_\coqdocindent{0.50em}
\_\coqdocindent{0.50em}
\_\coqdocindent{0.50em}
\_\coqdocindent{0.50em}
i)\^{}\coqdocindent{0.50em}
@\coqdocindent{0.50em}
\_).\coqdoceol
\coqdocindent{1.00em}
\begin{coqdoccomment}
\coqdocindent{0.50em}
this\coqdocindent{0.50em}
is\coqdocindent{0.50em}
just\coqdocindent{0.50em}
voodoo,\coqdocindent{0.50em}
basically\coqdocindent{0.50em}
\end{coqdoccomment}
\coqdoceol
\coqdocindent{1.00em}
f\_ap.\coqdocindent{0.50em}
symmetry.\coqdocindent{0.50em}
destruct\coqdocindent{0.50em}
(H\coqdocindent{0.50em}
b\coqdocindent{0.50em}
c).\coqdocindent{0.50em}
rewrite\coqdocindent{0.50em}
<-\coqdocindent{0.50em}
(eisretr\coqdocindent{0.50em}
i).\coqdocindent{0.50em}
simpl.\coqdocindent{0.50em}
f\_ap.\coqdoceol
\coqdoceol
\coqdocindent{1.00em}
intro\coqdocindent{0.50em}
iso.\coqdocindent{0.50em}
\coqdoceol
\coqdocindent{1.00em}
apply\coqdocindent{0.50em}
path\_isomorphic.\coqdocindent{0.50em}
simpl.\coqdoceol
\coqdocindent{1.00em}
destruct\coqdocindent{0.50em}
iso\coqdocindent{0.50em}
as\coqdocindent{0.50em}
[\coqdocvar{iso} \coqdocvar{iso\_comm}] [[\coqdocvar{iso\_inv} \coqdocvar{iso\_inv\_comm}] \coqdocvar{r\_inv} \coqdocvar{l\_inv}].\coqdoceol
\coqdocindent{1.00em}
apply\coqdocindent{0.50em}
path\_morphism.\coqdoceol
\coqdocindent{1.00em}
simpl\coqdocindent{0.50em}
in\coqdocindent{0.50em}
*.\coqdoceol
\coqdocindent{1.00em}
path\_induction.\coqdocindent{0.50em}
simpl\coqdocindent{0.50em}
in\coqdocindent{0.50em}
*.\coqdoceol
Admitted.\coqdoceol
\end{coqdoccomment}
\coqdoceol
\coqdocemptyline
\coqdocemptyline
\coqdocnoindent
\coqdockw{End} \coqref{Ch09}{\coqdocmodule{my\_slice}}.\coqdoceol
\coqdocemptyline
\end{coqdoccode}
\exer{9.2}{334} 
For any set $X$, prove that the slice category $\uset/X$ is equivalent to the
functor category $\uset^{X}$, where in the latter case we regard $X$ as a
discrete category.


 \soln
Because $\uset$ is a category, so are $\uset/X$ (by the previous exercise) and
$\uset^{X}$ (by Theorem 9.2.5).  So it suffices to show that there is an
isomorphism of categories $F : A \to B$.  Define $F(f) \defeq \hfib{f}{-}$ and
\begin{align*}
  F(h) \defeq (a, p) \mapsto \left(\fst(h)(a), \happly(\snd(h)^{-1},a) \ct
  p)\right) 
\end{align*}
for $f : \uset/X$ and $h : \hom_{\uset/X}(f, g)$.  In topological terms, the
bundle $f : A \to X$ gets set to its fiber map, and the bundle map $h$ gets set
to the pullback.  


To show that $F_{0} : (\uset/X)_{0} \to (\uset^{X})_{0}$ is an equivalence of
types, we need a quasi-inverse.  Suppose that $G : \uset^{X}$, and define
$F^{-1}_{0}(G) \defeq \left(\sm{x:X}G(x), \fst\right) : \uset/X$.  That is, we
take the disjoint union of all the fibers to reconstruct the bundle.  Now to
show that this is a quasi-inverse.  Suppose that $(A, f) : \uset/X$.  Then
\[
  F^{-1}(F(A, f)) 
  \equiv \left(\sm{x:X}\sm{a:A}(f(a) = x), \fst\right)
\]
Now, we have
\[
  e : 
  \sm{x:X} \sm{a:A} (f(a) = x)
  \eqvsym
  \sm{a:A} \sm{x:X} (f(a) = x)
  \eqvsym
  \sm{a:A} \unit
  \eqvsym
  A
\]
So $\ua(e) : \fst(F^{-1}(F(A, f))) = A$.  Now we must show
\[
  \transfib{\hom_{\uset}(-, X)}{\ua(e)}{\fst} = f
\]                      
Suppose that $a:A$.  Applying the left side to $a$, we get
\begin{align*}
  \left(\transfib{\hom_{\uset}(-,X)}{\ua(e)}{\fst}\right)(a)
  &=
  \transfib{X}{\ua(e)}{\fst(\transfib{X \mapsto X}{\ua(e)^{-1}}{a})}
  \\&=
  \fst(\transfib{X \mapsto X}{\ua(e)^{-1}}{a})
  \\&=
  \fst(e^{-1}(a))
  \\&=
  \fst(f(a), (a, \refl{f(a)}))
  \\&=
  f(a)
\end{align*}
So by function extensionality, our second components are equal, and
$F^{-1}(F(A, f)) = (A, f)$.


Suppose instead that $G : \uset^{X}$.  Then
\[
  F(F^{-1}(G)) \equiv \hfib{\fst}{-}
\]
which we must show is equal to $G$.  Identity of functors is determined by
identity of the functions $X \to \uset$ and the corresponding functions on
hom-sets.  For the first, by function extensionality it suffices to show that
\[
  G(x)
  =
  \sm{z : \sm{x:X}G(x)}(\fst(z) = x)
\]
for any $x : X$.  Any by univalence, it suffices to show that 
\[
  G(x)
  \eqvsym
  \sm{z : \sm{x':X}G(x')}(\fst(z) = x)
\]
which is true:
\begin{align*}
  \sm{z : \sm{x':X}G(x')}(\fst(z) = x)
  &\eqvsym \sm{x':X}\sm{g':G(x')}(x'=x)
  \\&\eqvsym \sm{x':X}\sm{g:G(x)}(x'=x)
  \\&\eqvsym \sm{g:G(x)}\sm{x':X}(x'=x)
  \\&\eqvsym \sm{g:G(x)}\unit
  \\&\eqvsym G(x)
\end{align*}
For the function on the hom-set, we again use function extensionality.  Let $h
: \prd{x:X} \hfib{\fst}{x} = G(x)$ be the path we just constructed, and let
$x, x' : X$.  We need to show that
\[
  (h(x), h(x'))_{*}\hfib{\fst}{-} = G 
  : \hom_{X}(x, x') \to \hom_{\uset}(G(x), G(x'))
\]
Since $X$ is a discrete category, we have $\hom_{X}(x, x') \defeq (x = x')$.
By function extensionality, it suffices to show that for any $p : x = x'$,
\[
  (h(x), h(x'))_{*}\hfib{\fst}{p} = G(p)
\]
By path induction, it suffices to consider the case where $x \equiv x'$ and $p
\equiv \refl{x}$.  Then we have
\[
  (h(x), h(x))_{*}\hfib{\fst}{\refl{x}}
  =
  (h(x), h(x))_{*}1
  = 
  \idtoiso(h(x)) \circ 1 \circ \idtoiso(h(x))^{-1}
  =
  1
  =
  G(\refl{x})
\]
So $F(F^{-1}(G)) = G$.  Thus $F$ and $F^{-1}$ are quasi-inverses, so
\[
  F : \uset/X \eqvsym \uset^{X}
\]


 \exer{9.3}{334} 
Prove that a functor is an equivalence of categories if and only if it is a
\emph{right} adjoint whose unit and counit are isomorphisms.


 \soln
A functor $F : A \to B$ is a right adjoint if there exist



\begin{itemize}
\item  A functor $G : B \to A$,

\item  A natural transformation $\epsilon : GF \to 1_{A}$

\item  A natural transformation $\eta : 1_{B} \to FG$

\item  $(F\epsilon)(\eta F) = 1_{F}$

\item  $(\epsilon G)(G\eta) = 1_{G}$

\end{itemize}


\noindent
Suppose that $F : \eqv{A}{B}$.  Then it is a left adjoint, so we have a
functor $G : B \to A$, a unit $\eta : 1_{A} \to GF$, a counit $\epsilon : FG
\to 1_{B}$, and the triangle identities $(\epsilon F)(F \eta) = 1_{F}$ and $(G
\epsilon)(\eta G) = 1_{G}$.  Furthermore, $\eta : 1_{A} \cong GF$ and $\epsilon
: FG \cong 1_{B}$.  Thus there are natural transformations $\eta^{-1} : GF \to
1_{A}$ and $\epsilon^{-1} : 1_{B} \to FG$, and we have for all $a : A$
\[
  \left((F\eta)(F\eta^{-1})\right)_{a}
  =
  (F\eta)_{a}(F\eta^{-1})_{a}
  =
  F(\eta_{a}) \circ F(\eta^{-1}_{a})
  =
  F(\eta_{a} \circ \eta^{-1}_{a})
  =
  F(1_{a})
  =
  1_{Fa}
  =
  (1_{F})_{a}
\]
and
\[
  \left((\epsilon^{-1}F)(\epsilon F)\right)_{a}
  =
  (\epsilon^{-1}F)_{a}(\epsilon F)_{a}
  =
  \epsilon^{-1}_{Fa} \circ \epsilon_{Fa}
  =
  1_{Fa}
  =
  (1_{F})_{a}
\]
So by function extensionality $(F\eta)(F\eta^{-1}) = 1_{F}$ and
$(\epsilon^{-1}F)(\epsilon F) = 1_{F}$.  Thus
\begin{align*}
  (F\eta^{-1})(\epsilon^{-1}F)
  &=
  (\epsilon F)(F\eta)
  (F\eta^{-1})(\epsilon^{-1}F)
  (\epsilon F)(F \eta)
  =
  (\epsilon F)
  (F \eta)
  =
  1_{F}
  \\
  (\eta^{-1}G)(G\epsilon^{-1})
  &=
  (G\epsilon)(\eta G)
  (\eta^{-1}G)(G\epsilon^{-1})
  (G\epsilon)(\eta G)
  =
  (G\epsilon)
  (\eta G)
  =
  1_{G}
\end{align*}
So $(G, \eta^{-1}, \epsilon^{-1})$ makes $F$ a right adjoint.


Suppose instead that $F$ is a right adjoint by $(G, \epsilon, \eta)$, and that
$\eta$ and $\epsilon$ are isomorphisms.  To show that it is an equivalence of
categories, we have to show that it is a left adjoint with isomorphisms for the
unit and counit.  We have $G : B \to A$, $\epsilon^{-1} : 1_{A} \to GF$, and
$\eta^{-1} : FG \to 1_{B}$, and these natural transformations are isos, meaning
that just need to show that the triangle identity holds here.  We have
\begin{align*}
  (\eta^{-1}F)(F\epsilon^{-1})
  &=
  (F\epsilon)(\eta F)
  (\eta^{-1}F)(F\epsilon^{-1})
  (F\epsilon)(\eta F)
  =
  (F\epsilon)
  (\eta F)
  =
  1_{F}
  \\
  (G \eta^{-1})(\epsilon^{-1} G)
  &=
  (\epsilon G)(G \eta)
  (G \eta^{-1})(\epsilon^{-1} G)
  (\epsilon G)(G \eta)
  =
  (\epsilon G)(G \eta)
  =
  1_{G}
\end{align*}
and so $(G, \eta^{-1}, \epsilon^{-1})$ makes $F$ a left adjoint, hence an
equivalence of categories.


 \exer{9.4}{334} 
Define the notion of pre-2-category.  Show that precategories, functors, and
natural transformations as defined in \S9.2 form a pre-2-category.  Similarly,
define a pre-bicategory by replacing the equalities (such as those in Lemmas
9.2.9 and 9.2.11) with natural isomorphisms satisfying analogous coherence
conditions.  Define a function from pre-2-categories to pre-bicategories, and
show that it becomes an equivalence when restricted and corestricted to those
whose hom-precategories are categories.


 \soln
A pre-2-category $A$ consists of
\begin{enumerate}
  \item A type $A_{0}$ of 0-cells
  \item For all $a, b : A$, a precategory $C(a, b)$ whose objects are 1-cells
  and whose morphisms are 2-cells.  Composition in this precategory is denoted
  $\circ_{1}$ and called vertical composition.
  \item For all $a : A$, an object $1_{a} : C(a, a)$.
  \item For all $a, b, c : A$, a functor $\circ_{0} : C(b, c) \to C(a, b) \to
  C(a, c)$ called horizontal composition, which is associative and takes
  $1_{A}$ and $1_{1_{A}}$ as identities, for which the analogues of Lemmata
  9.2.10 and 9.2.11 hold.
\end{enumerate}


To check that precategories, functors, and natural transformations form a
pre-2-category, let precategories form the type of 0-cells.  To each pair of
precategories $A, B$, we have the functor precategory $B^{A}$, given by
definition 9.2.3.  For all precatgories $A$, let $1_{A} : A^{A}$ be the
identity functor.  For horizontal composition we have composition of functors,
which is associative by Lemma 9.2.9 and takes $1_{A}$ as an identity by Lemma
9.2.11.  So we're done.


A pre-bicategory $A$ consists of
\begin{enumerate}
  \item A type $A_{0}$ of 0-cells
  \item For all $a, b : A$, a precategory $C(a, b)$ whose objects are 1-cells
  and whose morphisms are 2-cells.  Composition in this precategory is denoted
  $\circ_{1}$ and called vertical composition.
  \item For all $a : A$, an object $1_{a} : C(a, a)$.
  \item For all $a, b, c : A$, a functor $\circ_{0} : C(b, c) \to C(a, b) \to
  C(a, c)$ called horizontal composition.
  \item For all $a, b : A$ and $f : C(a, b)$, isomorphisms $\rho_{f} : f
  \circ_{0} 1_{a} \cong f$ and $\lambda_{f} : 1_{b} \circ_{0} f \cong f$.
  \item For all $a, b, c, d : A$, $f : C(a, b)$, $g : C(b, c)$, and $h : C(c,
  d)$, an isomorphism $\alpha_{h, g, f} : (h \circ_{0} g) \circ_{0} f \cong h
  \circ_{0} (g \circ_{0} f)$.
  \item Mac Lane's pentagon commutes.  For all $a, b, c, d, e : A$
  $f : C(a, b)$, $g : C(b, c)$, $h : C(c, d)$, and $i : C(d, e)$,
  \[\xymatrix@=1in{
    ((i \circ_{0} h) \circ_{0} g) \circ_{0} f
    \ar[r]^{\alpha_{i, h, g}\circ_{0} 1_{f}}
    \ar[d]_{\alpha_{i \circ_{0} h, g, f}}
    & 
    (i \circ_{0} (h \circ_{0} g)) \circ_{0} f 
    \ar[r]^{\alpha_{i, h \circ_{0} g, f}}  
    &
    i \circ_{0} ((h \circ_{0} g) \circ_{0} f)
    \ar[d]^{1_{i} \circ_{0} \alpha_{h, g, f}} \\
    (i \circ_{0} h) \circ_{0} (h \circ_{0} f)
    \ar[rr]_{\alpha_{i, h, g \circ_{0} f}}
    & & 
    i \circ_{0} (h \circ_{0} (g \circ_{0} f))
  }\]
  \item Identity and associativity commute:
  \[\xymatrix{
  (g \circ_{0} 1_{b}) \circ_{0} f
  \ar[rr]^{\alpha_{g, 1_{b}, f}}
  \ar[rd]_{\rho_{g}\circ_{0} 1_{f}}
  & &
  g \circ_{0} (1_{b} \circ_{0} f)
  \ar[dl]^{1_{b} \circ_{0} \lambda_{f}} \\
  & g \circ_{0} f
  }\]
\end{enumerate}
The last two conditions can also be seen as Lemmata 9.2.10 and 9.2.11 with the
equalities replaced by isomorphisms.


Given a pre-2-category $A$, we can obtain a pre-bicategory using $\idtoiso$.
The type of 0-cells is the same, as are the precategories $C(a, b)$, the
identities $1_{a}$, and the functor $\circ_{0}$.  For a pre-2-category we have
$\bar{\rho}_{f} : f \circ_{0} 1_{a} = f$ and $\bar{\lambda}_{f} : 1_{b}
\circ_{0} f = f$, so $\rho_{f} \defeq \idtoiso(\bar{\rho}_{f}) : f \circ_{0}
1_{a} \cong f$ and $\lambda_{f} \defeq \idtoiso(\bar{\lambda}_{f}) : 1_{b}
\circ_{0} f \cong f$.  Similarly, we have $\bar{\alpha}_{h, g, f} : (h
\circ_{0} g) \circ_{0} f = h \circ_{0} (g \circ_{0} f)$ and $\alpha_{h, g, f}
\defeq \idtoiso(\bar{\alpha}_{h, g, f}) : (h
\circ_{0} g) \circ_{0} f \cong h \circ_{0} (g \circ_{0} f)$.  The coherence
conditions follow from Lemmata 9.2.10 and 9.2.11; since $\idtoiso(p \ct q) =
\idtoiso(q) \circ \idtoiso(p)$, applying $\idtoiso$ everywhere in these lemmata
give the commuting pentagon and triangle.


Finally, restrict and corestrict to pre-2-categories and pre-bicategories whose
hom-precategories are categories.  We need to construct a quasi-inverse to the
map just given, which will basically be systematic application of $\isotoid$.
The type of 0-cells, $C(a, b)$, $1_{a}$, and $\circ_{0}$ remain the same.  From
(v) we have $\rho_{f} : f \circ_{0} 1_{a} \cong f$, hence $\bar{\rho}_{f}
\defeq \isotoid(\rho_{f}) : f \circ_{0} 1_{a} = f$, and $\lambda_{f} : 1_{b}
\circ_{0} f \cong f$, hence $\bar{\lambda_{f}} \defeq \isotoid(\lambda_{f}) :
1_{b} \circ_{0} f = f$.  Applying $\isotoid$ to everything in sight gives the
rest of condition (iv) on pre-2-categories.


These two processes are clearly quasi-inverses, which follows from the fact
that $\idtoiso$ is an equivalence.


 \exer{9.5}{334} 
Define a 2-category to be a pre-2-category satisfying a condition analogous to
that of Definition 9.1.6.  Verify that the pre-2-category of categories $\ucat$
is a 2-category.  How much of this chapter can be done internally to an
arbitrary 2-category?


 \exer{9.6}{334}  \exer{9.7}{334}  \exer{9.8}{334} 

 \exer{9.9}{334} 
Prove that a function $X \to Y$ is an equivalence if and only if its image in
the homotopy category of Example 9.9.7 is an isomorphism.  Show that the type
of objects of this category is $\brck{\UU}_{1}$.


 \exer{9.10}{335}  \exer{9.11}{335} 

 \exer{9.12}{335} 
Let $X$ and $Y$ be sets and $p : Y \to X$ a surjection
\begin{enumerate}
  \item Define, for any precategory $A$, the category $\mathrm{Desc}(A, p)$ of
  descent data in $A$ relative to $p$.
  \item  Show that any precategory $A$ is a prestack for $p$, i.e.~the
  canonical functor $A^{X} \to \mathrm{Desc}(A, p)$ is fully faithful.
  \item Show that if $A$ is a category, then it is a stack for $p$,
      i.e.~$A^{X} \to \mathrm{Desc}(A, p)$ is an equivalence.
  \item Show that the statement ``every strict category is a stack for every
      surjection of sets'' is equivalent to the axiom of choice.
\end{enumerate}
\begin{coqdoccode}
\end{coqdoccode}
