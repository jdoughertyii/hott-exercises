\begin{coqdoccode}
\end{coqdoccode}
\section{Homotopy $n$-types}



 \exer{7.1}{250} 
(i) Use Theorem 7.2.2 to show that if $\brck{A} \to A$ for every type $A$, then
    every type is a set.
(ii) Show that if every surjective function (purely) splits, i.e.~if
    $\prd{b:B}\brck{\hfib{f}{b}} \to \prd{b:B}\hfib{f}{b}$ for every $f : A \to
    B$, then every type is a set.


 \soln
(i)  Suppose that $f : \prd{A:\UU}\brck{A} \to A$, and let $X$ be some type. To
apply Theorem 7.2.2, we need a reflexive mere relation on $X$ that implies
identity.  Let $R(x, y) \defeq \brck{x = y}$, which is clearly a reflexive mere
relation.  Moreoever, it implies the identity, since for any $x, y : X$ we have
\[
  f(x = y) : R(x, y) \to (x = y)
\]
So by Theorem 7.2.2, $X$ is a set.




 \exer{7.2}{250} 
Express $\Sn^{2}$ as a colimit of a diagram consisting entirely of copies of
$\unit$.


 \soln
Recall that we can define $\Sn^{2} \defeq \susp\Sn^{1} \defeq \susp\susp\Sn^{0}
\defeq \susp\susp\susp\bool$.  Then note that $\eqv{\bool}{\unit + \unit}$, and
that $\susp{A}$ is the pushout of the span $\unit \leftarrow A \to \unit$.  So
we have a diagram
\[\xymatrix{
  \unit + \unit \ar[r] \ar[d] & \unit \ar[d] & \\
  \unit \ar[r] & \Sn^{1} \ar[d] \ar[r] & \unit \ar[d] \\
  & \unit \ar[r] & \Sn^{2}
}\]
where both squares are pushouts.  And, since the coproduct is the colimit
of the discrete diagram with two items, we can write
\[\xymatrix{
               & \unit \ar[d] & & \\
  \unit \ar[r] & \bool \ar[r] \ar[d] & \unit \ar[d] & \\
  & \unit \ar[r] & \Sn^{1} \ar[d] \ar[r] & \unit \ar[d] \\
  & & \unit \ar[r] & \Sn^{2}
}\]
where now every object that isn't a $\unit$ is a colimit.


 \exer{7.3}{250} \begin{coqdoccode}
\coqdocemptyline
\coqdocnoindent
\coqdockw{Inductive} \coqdef{Ch07.W tree}{W\_tree}{\coqdocinductive{W\_tree}} (\coqdocvar{A} : \coqdockw{Type}) (\coqdocvar{B} : \coqdocvariable{A} \coqexternalref{:type scope:x '->' x}{http://coq.inria.fr/distrib/8.4pl4/stdlib/Coq.Init.Logic}{\coqdocnotation{\ensuremath{\rightarrow}}} \coqdockw{Type}) : \coqdockw{Type} :=\coqdoceol
\coqdocindent{1.00em}
\ensuremath{|} \coqdef{Ch07.sup}{sup}{\coqdocconstructor{sup}} : \coqdockw{\ensuremath{\forall}} \coqdocvar{a} : \coqdocvar{A}, \coqexternalref{:type scope:x '->' x}{http://coq.inria.fr/distrib/8.4pl4/stdlib/Coq.Init.Logic}{\coqdocnotation{(}}\coqdocvar{B} \coqdocvariable{a} \coqexternalref{:type scope:x '->' x}{http://coq.inria.fr/distrib/8.4pl4/stdlib/Coq.Init.Logic}{\coqdocnotation{\ensuremath{\rightarrow}}} \coqref{Ch07.W tree}{\coqdocinductive{W\_tree}} \coqdocvar{A} \coqdocvar{B}\coqexternalref{:type scope:x '->' x}{http://coq.inria.fr/distrib/8.4pl4/stdlib/Coq.Init.Logic}{\coqdocnotation{)}} \coqexternalref{:type scope:x '->' x}{http://coq.inria.fr/distrib/8.4pl4/stdlib/Coq.Init.Logic}{\coqdocnotation{\ensuremath{\rightarrow}}} \coqref{Ch07.W tree}{\coqdocinductive{W\_tree}} \coqdocvar{A} \coqdocvar{B}.\coqdoceol
\coqdocemptyline
\end{coqdoccode}
\exer{7.4}{250}  \exer{7.5}{250}  \exer{7.6}{250}  \exer{7.7}{250}  \exer{7.8}{250}  \exer{7.9}{251}  \exer{7.10}{251}  \exer{7.11}{251} 

 \exer{7.12}{251} 
Show that $X \mapsto (\lnot \lnot X)$ is a modality.


 \soln
We must show that there are
\begin{enumerate}
  \item functions $\eta^{\lnot \lnot}_{A} : A \to \lnot\lnot A$ for every type
  $A$,
  \item for every $A : \UU$ and every family $B : \lnot\lnot A \to \UU$, a
  function
  \[
    \ind{\lnot\lnot} : 
    \left(\prd{a : A}\lnot\lnot(B(\eta^{\lnot\lnot}_{A}(a)))\right)
    \to
    \prd{z : \lnot\lnot A}\lnot\lnot(B(z))
  \]
  \item A path $\ind{\lnot\lnot}(f)(\eta^{\lnot\lnot}_{A}(a)) = f(a)$ for each
  $f : \prd{a : A}\lnot\lnot(B(\eta^{\lnot\lnot}_{A}(a)))$, and
  \item For any $z, z' : \lnot \lnot A$, the function $\eta^{\lnot\lnot}_{z =
  z'} : (z = z') \to \lnot\lnot(z = z')$ is an equivalence.
\end{enumerate}
Some of this has already been done.  The functions $\eta^{\lnot\lnot}_{A}$ were
given in Exercise 1.12; recall:
\[
  \eta^{\lnot\lnot}_{A} \defeq \lam{a}{f}f(a)
\]
Now suppose that we have an element $f$ of the antecedent of
$\ind{\lnot\lnot}$, and let $z : \lnot\lnot A$.  Suppose furthermore
that $g : \lnot B(z)$; we derive a contradiction.  It suffices to
construct an element of $\lnot A$, for then we can apply $z$ to obtain
an element of $\emptyt$.  So suppose that $a : A$.  Then 
\[
  f(a) : \lnot \lnot B(\eta^{\lnot\lnot}_{A}(a))
\]
but $\lnot X$ is a mere proposition for any type $X$, so $z =
\eta^{\lnot\lnot}_{A}(a)$, and transporting gives an element of $\lnot \lnot
B(z)$.  Applying this to $g$ gives a contradiction.


For (iii), suppose that $a : A$ and $f : \prd{a :
A}\lnot\lnot(B(\eta^{\lnot\lnot}_{A}(a)))$.  Then $f(a) : \lnot \lnot
(B(\eta^{\lnot\lnot}_{A}(a)))$, making this type contractible, giving a
canonical path $\ind{\lnot\lnot}(f)(\eta^{\lnot\lnot}_{A}(a)) = f(a)$.


Finally, for (iv), let $z, z' : \lnot\lnot A$.  Since $\lnot\lnot A$ is a mere
proposition, so is $z = z'$.  $\lnot\lnot(z = z')$ is also a mere proposition,
so if there is an arrow $\lnot\lnot(z = z') \to (z = z')$, then it is
automatically a quasi-inverse to $\eta^{\lnot\lnot}_{z = z'}$.  Such an arrow
is immediate from the fact that $\lnot \lnot A$ is a mere proposition.


The definition of a modalities in HoTT/HoTT is different from the one
in the book, in order to get better computation rules.
\begin{coqdoccode}
\coqdocemptyline
\coqdocnoindent
\coqdockw{Module} \coqdef{Ch07.Ex12}{Ex12}{\coqdocmodule{Ex12}}.\coqdoceol
\coqdocemptyline
\coqdocnoindent
\coqdockw{Definition} \coqdef{Ch07.Ex12.notnot modality}{notnot\_modality}{\coqdocdefinition{notnot\_modality}} `\{\coqdocclass{Funext}\} : \coqdocclass{Modality}.\coqdoceol
\coqdocnoindent
\coqdockw{Proof}.\coqdoceol
\coqdocindent{1.00em}
\coqdoctac{refine} (\coqdocdefinition{Build\_Modality\_easy} (\coqdockw{fun} \coqdocvar{A} \ensuremath{\Rightarrow} \coqdocnotation{\ensuremath{\lnot}} \coqdocnotation{\ensuremath{\lnot}} \coqdocvariable{A}) \coqdocvar{\_} \coqdocvar{\_} \coqdocvar{\_} \coqdocvar{\_}).\coqdoceol
\coqdocindent{1.00em}
- \coqdoctac{intros} \coqdocvar{A} \coqdocvar{a} \coqdocvar{f}. \coqdoctac{apply} (\coqdocvar{f} \coqdocvar{a}).\coqdoceol
\coqdocindent{1.00em}
- \coqdoctac{intros} \coqdocvar{A} \coqdocvar{B} \coqdocvar{f} \coqdocvar{z} \coqdocvar{g}.\coqdoceol
\coqdocindent{2.00em}
\coqdoctac{apply} \coqdocvar{z}. \coqdoctac{intro} \coqdocvar{a}.\coqdoceol
\coqdocindent{2.00em}
\coqdoctac{refine} ((\coqdocdefinition{transport} (\coqdockw{fun} \coqdocvar{x} \ensuremath{\Rightarrow} \coqdocnotation{\ensuremath{\lnot}} \coqdocnotation{\ensuremath{\lnot}} \coqdocnotation{(}\coqdocvar{B} \coqdocvariable{x}\coqdocnotation{)}) \coqdoceol
\coqdocindent{11.50em}
(@\coqdoclemma{allpath\_hprop} \coqdocvar{\_} \coqdocvar{\_} \coqdocvar{\_} \coqdocvar{\_})\coqdoceol
\coqdocindent{11.50em}
(\coqdocvar{f} \coqdocvar{a})) \coqdocvar{g}).\coqdoceol
\coqdocindent{1.00em}
- \coqdoctac{intros} \coqdocvar{A} \coqdocvar{B} \coqdocvar{f} \coqdocvar{a}. \coqdoctac{apply} \coqdoclemma{allpath\_hprop}.\coqdoceol
\coqdocindent{1.00em}
- \coqdoctac{intros} \coqdocvar{A} \coqdocvar{z} \coqdocvar{z'}.\coqdoceol
\coqdocindent{2.00em}
\coqdoctac{refine} (\coqdocdefinition{isequiv\_adjointify} \coqdocvar{\_} \coqdocvar{\_} \coqdocvar{\_} \coqdocvar{\_}).\coqdoceol
\coqdocindent{2.00em}
\ensuremath{\times} \coqdoctac{intros} \coqdocvar{f}. \coqdoctac{apply} \coqdoclemma{allpath\_hprop}.\coqdoceol
\coqdocindent{2.00em}
\ensuremath{\times} \coqdoctac{intro} \coqdocvar{p}. \coqdoctac{apply} \coqdocdefinition{path\_arrow}; \coqdoctac{intro} \coqdocvar{f}. \coqdocvar{contradiction}.\coqdoceol
\coqdocindent{2.00em}
\ensuremath{\times} \coqdoctac{intro} \coqdocvar{p}. \coqdoctac{apply} \coqdoclemma{allpath\_hprop}.\coqdoceol
\coqdocnoindent
\coqdockw{Defined}.\coqdoceol
\coqdocemptyline
\coqdocnoindent
\coqdockw{End} \coqref{Ch07}{\coqdocmodule{Ex12}}.\coqdoceol
\coqdocemptyline
\coqdocemptyline
\end{coqdoccode}
\exer{7.13}{251} 
Let $P$ be a mere proposition
\begin{enumerate}
\item Show that $X \mapsto (P \to X)$ is a left exact modality.
\item Show that $X \mapsto (P * X)$ is a left exact modality, where $*$ denotes
    the join.
\end{enumerate}


 \soln
A left exact modality is one which preserves pullbacks as well as finite
products.  That is, if $A \times _{C} B$ is a pullback, then so is
$\modal (A \times_{C} B)$, which is to say that for all $X$,
\[
  (X \to \modal(A \times_{C} B))
  \eqvsym
  (X \to \modal A) \times_{X \to \modal C} (X \to \modal B)
\]
Likewise, if $A \times B$ is a product, then so is $\modal(A \times B)$;
i.e., for all $X$ we have
\[
  \eqv{
    (X \to \modal(A \times B)) 
  }{
    (X \to \modal A) \times (X \to \modal B)
  }
\]


\vspace{.1in}
\noindent
(i) Let $\modal X \defeq (P \to X)$, with $\eta^{\modal}_{A} : A \to
\modal A$ given by $\eta^{\modal}_{A}(a) \defeq \lam{p}a$.  To derive the
induction principle, suppose that $A : \UU$, $B : \modal A \to \UU$, $f :
\prd{a:A}\modal(B(\eta^{\modal}_{A}(a)))$, and $g : \modal A$.  We need to
construct an arrow $P \to B(g)$.  So suppose that $p : P$, so $g(p) : A$.  Then
$f(g(p)) : P \to B(\lam{p'}g(p))$.  But since $P$ is a mere proposition, by
function extensionality $\lam{p'}g(p) = g$, and we can transport the output of
$f(g(p))$ to obtain an element of $B(g)$, as required.


To demonstrate the computation rule, suppose that $f : \prd{a:A} P \to
B(\lam{p}a)$ and $a : A$.  Then $f(a) : P \to B(\lam{p}a)$ and $\ind{\modal
A}(f)(a) : P \to B(\lam{p}a)$.  So by function extensionality and the fact that
$P$ is a mere proposition, $\ind{\modal A}(f)(a) = f(a)$.


To show that $\eta^{\modal}_{z = z'}$ is an equivalence for all $f, g : \modal
A$, we first need an inverse.  Let $k : P \to (f = g)$, and suppose that $p :
P$.  Then $k(p) : f = g$, so $\lam{p}\happly(k(p)) : f \sim g$.  By function
extensionality, then, we have $f = g$.  These are clearly quasi-inverses.  If
$q : f = g$, then we want to show that
\[
  \funext(\lam{p}\happly(q, p)) = q
\]
which follows immediately from the fact that $\funext$ and $\happly$ are
inverses.  For the other direction, suppose that $k : P \to (f = g)$.  Then we
need to show that
\[
  \lam{p}\funext(\lam{p'}\happly(k(p'), p')) = k
\]
which we do by function extensionality.  Supposing that $p : P$, we need to
show that
\[
  \lam{p'}\happly(k(p'), p')) = \happly(k(p))
\]
and this follows from the fact that $P$ is a mere proposition, just as we
showed that $\lam{p'}g(p) = g$.


Now, to show that this modality is left exact, note that
\begin{align*}
 (X \to \modal(A \times_{C} B))
 &\equiv
 (X \to P \to A \times_{C} B)
 \\&\eqvsym
 ((X \times P) \to A \times_{C} B)
 \\&\eqvsym
 ((X \times P) \to A) \times_{(X \times P) \to C} ((X \times P) \to B)
 \\&\eqvsym
 (X \to P \to A) \times_{X \to P \to C} (X \to P \to B)
 \\&\equiv
 (X \to \modal A) \times_{X \to \modal C} (X \to \modal B)
\end{align*}
Using the cartesian closure adjunction.  Likewise,
\begin{align*}
  (X \to \modal(A \times B))
  &\equiv
  (X \to P \to A \times B)
  \\&\eqvsym
  ((X \times P) \to (A \times B))
  \\&\eqvsym
  ((X \times P) \to A)
  \times
  ((X \times P) \to B)
  \\&\eqvsym
  (X \to P \to A)
  \times
  (X \to P \to B)
  \\&\equiv
  (X \to \modal A)
  \times
  (X \to \modal B)
\end{align*}
By the universal property of the product.  So $\modal$ preserves both pullbacks
and finite products.


\vspace{.1in}
\noindent
(ii) Suppose now that $\modal X \defeq P * X$.  That is, suppose that $\modal X$ is
the pushout of the span $X \xleftarrow{\fst} X \times P \xrightarrow{\snd}
P$.  This is the higher inductive type presented by



\begin{itemize}
\item  a function $\inl : X \to P * X$

\item  a function $\inr : P \to P * X$

\item  for each $z : P \times X$ a path $\glue(z) : \inl(\fst(z)) = \inr(\snd(z))$

\end{itemize}
Note that if $P$ is contractible, then so is $\modal X$.  This is
straightforward: letting $p$ be the center of $P$, $\inl(p)$ is the center of
$\modal(X)$.  Using the induction principle for pushouts, we must first show
that for all $p' : P$, $\inl(p) = \inl(p')$, which follows from $P$'s
contractibility.  Next we need, for any $x : X$, $\inl(p) = \inr(x)$.  This
follows from $\glue((p, x)) : \inl(p) = \inr(x)$.  Finally, we must show that
for all $(p', x) : P \times X$, 
\[
  (\inl(p) = \inl(p')) = (\inl(p) = \inr(x))
\]
Which is pretty easy to show by some path algebra.


Now, suppose that $A : \UU$ and $B : \modal(A) \to \UU$, and let $f : \prd{a:A}
\modal(B(\eta^{\modal}_{A}(a)))$.  We use pushout induction to construct a map
$\prd{z:\modal A} \modal B(z)$.  When $z \equiv \inl(p)$, $P$ is inhabited, so
$\modal (B (z))$ is contractible and we can return its center.  When $z \equiv
\inr(a)$ then $f(a) : \modal(B(\inr(a)))$ is our element.  For the path,
suppose that $(p, x) : P \times X$.  Then $P$ is contractible, so all of its
higher path spaces are trivial and we're done.  This gives judgemental
computation rule, as well.


Finally, we need to show that $\eta^{\modal}_{z = z'}$ is an equivalence for
all $z, z' : \modal A$.  To define an inverse, suppose that $p : \modal(z =
z')$.  If $p \equiv \inl(p')$, then $P$ is contractible and so is $\modal A$,
so the higher path spaces are trivial and $z = z'$.  If $p \equiv \inr(q)$,
then $q : z = z'$ and we're done.  Finally, supposing that $(p, a) : P \times
A$, $P$ is contractible so $(z = z')$ is as well, so all the paths are trivial.
To show that these are quasi-inverses involves some minor path algebra.  Thus,
$\modal$ is a modality.


Now, to show that it is left exact.
\begin{coqdoccode}
\coqdocemptyline
\coqdocnoindent
\coqdockw{Lemma} \coqdef{Ch07.ap11 V}{ap11\_V}{\coqdoclemma{ap11\_V}} \{\coqdocvar{A} \coqdocvar{B} : \coqdockw{Type}\} \{\coqdocvar{f} \coqdocvar{g} : \coqdocvariable{A} \coqexternalref{:type scope:x '->' x}{http://coq.inria.fr/distrib/8.4pl4/stdlib/Coq.Init.Logic}{\coqdocnotation{\ensuremath{\rightarrow}}} \coqdocvariable{B}\} (\coqdocvar{p} : \coqdocvariable{f} \coqdocnotation{=} \coqdocvariable{g}) (\coqdocvar{h} : \coqdockw{\ensuremath{\forall}} \coqdocvar{x} \coqdocvar{y} : \coqdocvariable{A}, \coqdocvariable{x} \coqdocnotation{=} \coqdocvariable{y})\coqdoceol
\coqdocindent{3.00em}
(\coqdocvar{x} \coqdocvar{y} : \coqdocvariable{A})\coqdoceol
\coqdocindent{1.00em}
: \coqdocdefinition{ap11} \coqdocvariable{p}\coqdocnotation{\^{}} \coqdocnotation{(}\coqdocvariable{h} \coqdocvariable{x} \coqdocvariable{y}\coqdocnotation{)\^{}} \coqdocnotation{=} \coqdocnotation{(}\coqdocdefinition{ap11} \coqdocvariable{p} (\coqdocvariable{h} \coqdocvariable{x} \coqdocvariable{y})\coqdocnotation{)\^{}}.\coqdoceol
\coqdocnoindent
\coqdockw{Proof}.\coqdoceol
\coqdocindent{1.00em}
\coqdoctac{induction} \coqdocvar{p}.\coqdoceol
\coqdocindent{1.00em}
\coqdoctac{induction} (\coqdocvar{h} \coqdocvar{x} \coqdocvar{y}).\coqdoceol
\coqdocindent{1.00em}
\coqdoctac{reflexivity}.\coqdoceol
\coqdocnoindent
\coqdockw{Defined}.\coqdoceol
\coqdocemptyline
\coqdocnoindent
\coqdockw{Module} \coqdef{Ch07.Ex13}{Ex13}{\coqdocmodule{Ex13}}.\coqdoceol
\coqdocemptyline
\coqdocnoindent
\coqdockw{Require} \coqdockw{Import} \coqdoclibrary{ReflectiveSubuniverse}.\coqdoceol
\coqdocemptyline
\coqdocnoindent
\coqdockw{Section} \coqdef{Ch07.Ex13.OpenModality}{OpenModality}{\coqdocsection{OpenModality}}.\coqdoceol
\coqdocemptyline
\coqdocnoindent
\coqdockw{Context} (\coqdocvar{P} : \coqdocrecord{hProp}).\coqdoceol
\coqdocemptyline
\coqdocnoindent
\coqdockw{Definition} \coqdef{Ch07.Ex13.open modality}{open\_modality}{\coqdocdefinition{open\_modality}} `\{\coqdocclass{Funext}\} : \coqdocclass{Modality}.\coqdoceol
\coqdocnoindent
\coqdockw{Proof}.\coqdoceol
\coqdocindent{1.00em}
\coqdoctac{refine} (\coqdocdefinition{Build\_Modality\_easy} (\coqdockw{fun} \coqdocvar{X} \ensuremath{\Rightarrow} \coqdocvariable{P} \coqexternalref{:type scope:x '->' x}{http://coq.inria.fr/distrib/8.4pl4/stdlib/Coq.Init.Logic}{\coqdocnotation{\ensuremath{\rightarrow}}} \coqdocvariable{X}) \coqdocvar{\_} \coqdocvar{\_} \coqdocvar{\_} \coqdocvar{\_}).\coqdoceol
\coqdocindent{1.00em}
- \coqdoctac{intros} \coqdocvar{X} \coqdocvar{x} \coqdocvar{p}. \coqdoctac{apply} \coqdocvar{x}.\coqdoceol
\coqdocindent{1.00em}
- \coqdoctac{intros} \coqdocvar{A} \coqdocvar{B} \coqdocvar{f} \coqdocvar{g} \coqdocvar{p}.\coqdoceol
\coqdocindent{2.00em}
\coqdoctac{apply} (\coqdocdefinition{transport} \coqdocvar{B} (\coqdocdefinition{path\_arrow} (\coqdockw{fun} \coqdocvar{\_} : \coqdocvariable{P} \ensuremath{\Rightarrow} \coqdocvar{g} \coqdocvar{p}) \coqdocvar{g}\coqdoceol
\coqdocindent{18.00em}
(\coqdockw{fun} \coqdocvar{p'} : \coqdocvariable{P} \ensuremath{\Rightarrow} \coqdocdefinition{ap} \coqdocvar{g} (\coqdoclemma{allpath\_hprop} \coqdocvar{p} \coqdocvariable{p'})))).\coqdoceol
\coqdocindent{2.00em}
\coqdoctac{apply} (\coqdocvar{f} (\coqdocvar{g} \coqdocvar{p}) \coqdocvar{p}).\coqdoceol
\coqdocindent{1.00em}
- \coqdoctac{intros} \coqdocvar{A} \coqdocvar{B} \coqdocvar{f} \coqdocvar{a}.\coqdoceol
\coqdocindent{2.00em}
\coqdoctac{apply} \coqdocdefinition{path\_arrow}; \coqdoctac{intro} \coqdocvar{p}. \coqdoctac{simpl} \coqdoctac{in} *.\coqdoceol
\coqdocindent{2.00em}
\coqdocvar{path\_via} (\coqdocdefinition{transport} \coqdocvar{B} 1 (\coqdocvar{f} \coqdocvar{a} \coqdocvar{p})). \coqdocvar{f\_ap}.\coqdoceol
\coqdocindent{2.00em}
\coqdoctac{apply} \coqdocnotation{(}\coqdocdefinition{ap} \coqdocdefinition{apD10}\coqdocnotation{)\^{}-1}.\coqdoceol
\coqdocindent{2.00em}
\coqdoctac{apply} \coqdocdefinition{path\_forall}; \coqdoctac{intro} \coqdocvar{p'}.\coqdoceol
\coqdocindent{2.00em}
\coqdoctac{refine} (\coqdocnotation{(}\coqdocdefinition{apD10\_path\_arrow} \coqdocvar{\_} \coqdocvar{\_} \coqdocvar{\_} \coqdocvar{\_}\coqdocnotation{)} \coqdocnotation{@} \coqdocvar{\_}).\coqdoceol
\coqdocindent{2.00em}
\coqdoctac{apply} \coqdocdefinition{ap\_const}.\coqdoceol
\coqdocindent{1.00em}
- \coqdoctac{intros} \coqdocvar{A} \coqdocvar{z} \coqdocvar{z'}.\coqdoceol
\coqdocindent{2.00em}
\coqdoctac{refine} (\coqdocdefinition{equiv\_adjointify} \coqdocvar{\_} \coqdocvar{\_} \coqdocvar{\_} \coqdocvar{\_}).\coqdoceol
\coqdocindent{2.00em}
+ \coqdoctac{intro} \coqdocvar{f}. \coqdoctac{apply} \coqdocdefinition{path\_arrow}. \coqdoctac{intro} \coqdocvar{p}.\coqdoceol
\coqdocindent{3.00em}
\coqdoctac{apply} (\coqdocdefinition{ap10} (\coqdocvar{f} \coqdocvar{p}) \coqdocvar{p}).\coqdoceol
\coqdocindent{2.00em}
+ \coqdoctac{intro} \coqdocvar{f}. \coqdoctac{apply} \coqdocdefinition{path\_arrow}. \coqdoctac{intro} \coqdocvar{p}.\coqdoceol
\coqdocindent{3.00em}
\coqdocvar{path\_via} (\coqdocdefinition{path\_arrow} \coqdocvar{z} \coqdocvar{z'} (\coqdocdefinition{ap10} (\coqdocvar{f} \coqdocvar{p}))). \coqdocvar{f\_ap}.\coqdoceol
\coqdocindent{3.00em}
\ensuremath{\times} \coqdoctac{apply} \coqdocdefinition{path\_forall}. \coqdoctac{intro} \coqdocvar{p'}. \coqdocvar{f\_ap}. \coqdoctac{apply} (\coqdocdefinition{ap} \coqdocvar{f}). \coqdoctac{apply} \coqdoclemma{allpath\_hprop}.\coqdoceol
\coqdocindent{3.00em}
\ensuremath{\times} \coqdoctac{apply} \coqdocdefinition{eta\_path\_arrow}.\coqdoceol
\coqdocindent{2.00em}
+ \coqdoctac{intro} \coqdocvar{p}. \coqdoctac{apply} \coqdocdefinition{eta\_path\_arrow}.\coqdoceol
\coqdocnoindent
\coqdockw{Defined}.\coqdoceol
\coqdocemptyline
\coqdocnoindent
\coqdockw{Definition} \coqdef{Ch07.Ex13.preserves pullbacks}{preserves\_pullbacks}{\coqdocdefinition{preserves\_pullbacks}} (\coqdocvar{M} : \coqdocclass{Modality})\coqdoceol
\coqdocindent{1.00em}
:= \coqdockw{\ensuremath{\forall}} (\coqdocvar{A} \coqdocvar{B} \coqdocvar{C} : \coqdockw{Type}) (\coqdocvar{f} : \coqdocvariable{A} \coqexternalref{:type scope:x '->' x}{http://coq.inria.fr/distrib/8.4pl4/stdlib/Coq.Init.Logic}{\coqdocnotation{\ensuremath{\rightarrow}}} \coqdocvariable{C}) (\coqdocvar{g} : \coqdocvariable{B} \coqexternalref{:type scope:x '->' x}{http://coq.inria.fr/distrib/8.4pl4/stdlib/Coq.Init.Logic}{\coqdocnotation{\ensuremath{\rightarrow}}} \coqdocvariable{C}),\coqdoceol
\coqdocindent{3.50em}
\coqdocmethod{O} (\coqdocdefinition{pullback} \coqdocvariable{f} \coqdocvariable{g}) \coqdoceol
\coqdocindent{4.50em}
\coqdocnotation{\ensuremath{\eqvsym}} \coqdoceol
\coqdocindent{4.50em}
\coqdocdefinition{pullback} (\coqdocdefinition{O\_functor} \coqdocvariable{f}) (\coqdocdefinition{O\_functor} \coqdocvariable{g}).\coqdoceol
\coqdocemptyline
\coqdocnoindent
\coqdockw{Definition} \coqdef{Ch07.Ex13.preserves products}{preserves\_products}{\coqdocdefinition{preserves\_products}} (\coqdocvar{M} : \coqdocclass{Modality})\coqdoceol
\coqdocindent{1.00em}
:= \coqdockw{\ensuremath{\forall}} (\coqdocvar{A} \coqdocvar{B} : \coqdockw{Type}), \coqdocmethod{O} (\coqdocvariable{A} \coqexternalref{:type scope:x '*' x}{http://coq.inria.fr/distrib/8.4pl4/stdlib/Coq.Init.Datatypes}{\coqdocnotation{\ensuremath{\times}}} \coqdocvariable{B}) \coqdocnotation{\ensuremath{\eqvsym}} \coqdocnotation{(}\coqdocmethod{O} \coqdocvariable{A} \coqexternalref{:type scope:x '*' x}{http://coq.inria.fr/distrib/8.4pl4/stdlib/Coq.Init.Datatypes}{\coqdocnotation{\ensuremath{\times}}} \coqdocmethod{O} \coqdocvariable{B}\coqdocnotation{)}.\coqdoceol
\coqdocemptyline
\coqdocnoindent
\coqdockw{Class} \coqdef{Ch07.Ex13.IsLEM}{IsLEM}{\coqdocrecord{IsLEM}} (\coqdocvar{M} : \coqdocclass{Modality}) := \coqdoceol
\coqdocindent{1.00em}
\coqdef{Ch07.Ex13.BuildIsLEM}{BuildIsLEM}{\coqdocconstructor{BuildIsLEM}} \{\coqdoceol
\coqdocindent{3.00em}
\coqdef{Ch07.Ex13.lem pullbacks}{lem\_pullbacks}{\coqdocprojection{lem\_pullbacks}} : \coqref{Ch07.Ex13.preserves pullbacks}{\coqdocdefinition{preserves\_pullbacks}} \coqdocvariable{M} ;\coqdoceol
\coqdocindent{3.00em}
\coqdef{Ch07.Ex13.lem products}{lem\_products}{\coqdocprojection{lem\_products}} : \coqref{Ch07.Ex13.preserves products}{\coqdocdefinition{preserves\_products}} \coqdocvariable{M}\coqdoceol
\coqdocindent{2.00em}
\}.\coqdoceol
\coqdocemptyline
\coqdocnoindent
\coqdockw{Theorem} \coqdef{Ch07.Ex13.lem open}{lem\_open}{\coqdoclemma{lem\_open}} `\{\coqdocclass{Funext}\} : \coqref{Ch07.Ex13.IsLEM}{\coqdocclass{IsLEM}} \coqref{Ch07.Ex13.open modality}{\coqdocdefinition{open\_modality}}.\coqdoceol
\coqdocnoindent
\coqdockw{Proof}.\coqdoceol
\coqdocindent{1.00em}
\coqdoctac{refine} (\coqref{Ch07.Ex13.BuildIsLEM}{\coqdocconstructor{BuildIsLEM}} \coqdocvar{\_} \coqdocvar{\_} \coqdocvar{\_}).\coqdoceol
\coqdocemptyline
\coqdocindent{1.00em}
\begin{coqdoccomment}
\coqdocindent{0.50em}
preserves\coqdocindent{0.50em}
pullbacks\coqdocindent{0.50em}
\end{coqdoccomment}
\coqdoceol
\coqdocindent{1.00em}
- \coqdoctac{intros} \coqdocvar{A} \coqdocvar{B} \coqdocvar{C} \coqdocvar{f} \coqdocvar{g}.\coqdoceol
\coqdocindent{2.00em}
\coqdoctac{unfold} \coqdocdefinition{pullback}.\coqdoceol
\coqdocindent{2.00em}
\coqdoctac{simpl} (\coqdocmethod{O} \coqexternalref{:type scope:'x7B' x ':' x 'x26' x 'x7D'}{http://coq.inria.fr/distrib/8.4pl4/stdlib/Coq.Init.Specif}{\coqdocnotation{\{}}\coqdocvar{b} \coqexternalref{:type scope:'x7B' x ':' x 'x26' x 'x7D'}{http://coq.inria.fr/distrib/8.4pl4/stdlib/Coq.Init.Specif}{\coqdocnotation{:}} \coqdocvar{A} \coqexternalref{:type scope:'x7B' x ':' x 'x26' x 'x7D'}{http://coq.inria.fr/distrib/8.4pl4/stdlib/Coq.Init.Specif}{\coqdocnotation{\&}} \coqexternalref{:type scope:'x7B' x ':' x 'x26' x 'x7D'}{http://coq.inria.fr/distrib/8.4pl4/stdlib/Coq.Init.Specif}{\coqdocnotation{\{}}\coqdocvar{c} \coqexternalref{:type scope:'x7B' x ':' x 'x26' x 'x7D'}{http://coq.inria.fr/distrib/8.4pl4/stdlib/Coq.Init.Specif}{\coqdocnotation{:}} \coqdocvar{B} \coqexternalref{:type scope:'x7B' x ':' x 'x26' x 'x7D'}{http://coq.inria.fr/distrib/8.4pl4/stdlib/Coq.Init.Specif}{\coqdocnotation{\&}} \coqdocvar{f} \coqdocvar{b} \coqdocnotation{=} \coqdocvar{g} \coqdocvar{c}\coqexternalref{:type scope:'x7B' x ':' x 'x26' x 'x7D'}{http://coq.inria.fr/distrib/8.4pl4/stdlib/Coq.Init.Specif}{\coqdocnotation{\}\}}}).\coqdoceol
\coqdocindent{2.00em}
\coqdocvar{equiv\_via} \coqexternalref{:type scope:'x7B' x ':' x 'x26' x 'x7D'}{http://coq.inria.fr/distrib/8.4pl4/stdlib/Coq.Init.Specif}{\coqdocnotation{\{}}\coqdocvar{h} \coqexternalref{:type scope:'x7B' x ':' x 'x26' x 'x7D'}{http://coq.inria.fr/distrib/8.4pl4/stdlib/Coq.Init.Specif}{\coqdocnotation{:}} \coqdocvariable{P} \coqexternalref{:type scope:x '->' x}{http://coq.inria.fr/distrib/8.4pl4/stdlib/Coq.Init.Logic}{\coqdocnotation{\ensuremath{\rightarrow}}} \coqdocvar{A} \coqexternalref{:type scope:'x7B' x ':' x 'x26' x 'x7D'}{http://coq.inria.fr/distrib/8.4pl4/stdlib/Coq.Init.Specif}{\coqdocnotation{\&}} \coqexternalref{:type scope:'x7B' x ':' x 'x26' x 'x7D'}{http://coq.inria.fr/distrib/8.4pl4/stdlib/Coq.Init.Specif}{\coqdocnotation{\{}}\coqdocvar{j} \coqexternalref{:type scope:'x7B' x ':' x 'x26' x 'x7D'}{http://coq.inria.fr/distrib/8.4pl4/stdlib/Coq.Init.Specif}{\coqdocnotation{:}} \coqdocvariable{P} \coqexternalref{:type scope:x '->' x}{http://coq.inria.fr/distrib/8.4pl4/stdlib/Coq.Init.Logic}{\coqdocnotation{\ensuremath{\rightarrow}}} \coqdocvar{B} \coqexternalref{:type scope:'x7B' x ':' x 'x26' x 'x7D'}{http://coq.inria.fr/distrib/8.4pl4/stdlib/Coq.Init.Specif}{\coqdocnotation{\&}} \coqdocvar{f} \coqdocnotation{o} \coqdocvar{h} \coqdocnotation{\ensuremath{\sim}} \coqdocvar{g} \coqdocnotation{o} \coqdocvar{j}\coqexternalref{:type scope:'x7B' x ':' x 'x26' x 'x7D'}{http://coq.inria.fr/distrib/8.4pl4/stdlib/Coq.Init.Specif}{\coqdocnotation{\}\}}}.\coqdoceol
\coqdocindent{2.00em}
+ \coqdocvar{equiv\_via} \coqexternalref{:type scope:'x7B' x ':' x 'x26' x 'x7D'}{http://coq.inria.fr/distrib/8.4pl4/stdlib/Coq.Init.Specif}{\coqdocnotation{\{}}\coqdocvar{h} \coqexternalref{:type scope:'x7B' x ':' x 'x26' x 'x7D'}{http://coq.inria.fr/distrib/8.4pl4/stdlib/Coq.Init.Specif}{\coqdocnotation{:}} \coqdocvariable{P} \coqexternalref{:type scope:x '->' x}{http://coq.inria.fr/distrib/8.4pl4/stdlib/Coq.Init.Logic}{\coqdocnotation{\ensuremath{\rightarrow}}} \coqdocvar{A} \coqexternalref{:type scope:'x7B' x ':' x 'x26' x 'x7D'}{http://coq.inria.fr/distrib/8.4pl4/stdlib/Coq.Init.Specif}{\coqdocnotation{\&}} \coqdockw{\ensuremath{\forall}} \coqdocvar{p}, \coqexternalref{:type scope:'x7B' x ':' x 'x26' x 'x7D'}{http://coq.inria.fr/distrib/8.4pl4/stdlib/Coq.Init.Specif}{\coqdocnotation{\{}}\coqdocvar{b} \coqexternalref{:type scope:'x7B' x ':' x 'x26' x 'x7D'}{http://coq.inria.fr/distrib/8.4pl4/stdlib/Coq.Init.Specif}{\coqdocnotation{:}} \coqdocvar{B} \coqexternalref{:type scope:'x7B' x ':' x 'x26' x 'x7D'}{http://coq.inria.fr/distrib/8.4pl4/stdlib/Coq.Init.Specif}{\coqdocnotation{\&}} \coqdocvar{f} (\coqdocvar{h} \coqdocvariable{p}) \coqdocnotation{=} \coqdocvar{g} \coqdocvar{b}\coqexternalref{:type scope:'x7B' x ':' x 'x26' x 'x7D'}{http://coq.inria.fr/distrib/8.4pl4/stdlib/Coq.Init.Specif}{\coqdocnotation{\}\}}}.\coqdoceol
\coqdocindent{3.00em}
\ensuremath{\times} \coqdoctac{apply} \coqdoclemma{equiv\_inverse}. \coqdoctac{refine} (\coqdocdefinition{equiv\_sigT\_corect} \coqdocvar{\_} \coqdocvar{\_}).\coqdoceol
\coqdocindent{3.00em}
\ensuremath{\times} \coqdoctac{refine} (\coqdocdefinition{equiv\_functor\_sigma\_id} \coqdocvar{\_}). \coqdoctac{intro} \coqdocvar{h}.\coqdoceol
\coqdocindent{4.00em}
\coqdoctac{apply} \coqdoclemma{equiv\_inverse}. \coqdoctac{refine} (\coqdocdefinition{equiv\_sigT\_corect} \coqdocvar{\_} \coqdocvar{\_}).\coqdoceol
\coqdocindent{2.00em}
+ \coqdoctac{apply} \coqdocdefinition{equiv\_functor\_sigma\_id}. \coqdoctac{intro} \coqdocvar{h}.\coqdoceol
\coqdocindent{3.00em}
\coqdoctac{apply} \coqdocdefinition{equiv\_functor\_sigma\_id}. \coqdoctac{intro} \coqdocvar{j}.\coqdoceol
\coqdocindent{3.00em}
\coqdocvar{equiv\_via} (\coqdocvar{f} \coqdocnotation{o} \coqdocvar{h} \coqdocnotation{=} \coqdocvar{g} \coqdocnotation{o} \coqdocvar{j}). \coqdoctac{apply} \coqdocdefinition{equiv\_path\_arrow}.\coqdoceol
\coqdocindent{3.00em}
\coqdocvar{equiv\_via} (\coqdocnotation{(}@\coqdocmethod{O\_unit} \coqref{Ch07.Ex13.open modality}{\coqdocdefinition{open\_modality}} \coqdocvar{C}\coqdocnotation{)} \coqdocnotation{o} \coqdocnotation{(}\coqdocvar{f} \coqdocnotation{o} \coqdocvar{h}\coqdocnotation{)}\coqdoceol
\coqdocindent{8.50em}
\coqdocnotation{=}\coqdoceol
\coqdocindent{8.50em}
\coqdocnotation{(}@\coqdocmethod{O\_unit} \coqref{Ch07.Ex13.open modality}{\coqdocdefinition{open\_modality}} \coqdocvar{C}\coqdocnotation{)} \coqdocnotation{o} \coqdocnotation{(}\coqdocvar{g} \coqdocnotation{o} \coqdocvar{j}\coqdocnotation{)}).\coqdoceol
\coqdocindent{3.00em}
\coqdoctac{refine} (\coqdocdefinition{equiv\_adjointify} \coqdocvar{\_} \coqdocvar{\_} \coqdocvar{\_} \coqdocvar{\_}).\coqdoceol
\coqdocindent{4.00em}
\coqdoctac{intro} \coqdocvar{eq}. \coqdoctac{apply} \coqdocdefinition{path\_arrow}. \coqdoctac{intro} \coqdocvar{p}. \coqdoctac{apply} \coqdocdefinition{path\_arrow}. \coqdoctac{intro} \coqdocvar{p'}.\coqdoceol
\coqdocindent{4.00em}
\coqdoctac{apply} (\coqdocdefinition{ap10} \coqdocvar{eq} \coqdocvar{p}).\coqdoceol
\coqdocemptyline
\coqdocindent{4.00em}
\coqdoctac{intro} \coqdocvar{eq}. \coqdoctac{apply} \coqdocdefinition{path\_arrow}. \coqdoctac{intro} \coqdocvar{p}.\coqdoceol
\coqdocindent{4.00em}
\coqdoctac{apply} (\coqdocdefinition{apD10} (\coqdocdefinition{apD10} \coqdocvar{eq} \coqdocvar{p}) \coqdocvar{p}).\coqdoceol
\coqdocemptyline
\coqdocindent{4.00em}
\coqdoctac{intro} \coqdocvar{eq}.\coqdoceol
\coqdocindent{4.00em}
\coqdoctac{apply} \coqdocnotation{(}\coqdocdefinition{ap} \coqdocdefinition{apD10}\coqdocnotation{)\^{}-1}. \coqdoctac{apply} \coqdocdefinition{path\_forall}. \coqdoctac{intro} \coqdocvar{p}.\coqdoceol
\coqdocindent{4.00em}
\coqdoctac{refine} (\coqdocnotation{(}\coqdocdefinition{apD10\_path\_forall} \coqdocvar{\_} \coqdocvar{\_} \coqdocvar{\_} \coqdocvar{p}\coqdocnotation{)} \coqdocnotation{@} \coqdocvar{\_}).\coqdoceol
\coqdocindent{4.00em}
\coqdoctac{apply} \coqdocnotation{(}\coqdocdefinition{ap} \coqdocdefinition{apD10}\coqdocnotation{)\^{}-1}. \coqdoctac{apply} \coqdocdefinition{path\_forall}. \coqdoctac{intro} \coqdocvar{p'}.\coqdoceol
\coqdocindent{4.00em}
\coqdoctac{refine} (\coqdocnotation{(}\coqdocdefinition{apD10\_path\_forall} \coqdocvar{\_} \coqdocvar{\_} \coqdocvar{\_} \coqdocvar{p'}\coqdocnotation{)} \coqdocnotation{@} \coqdocvar{\_}).\coqdoceol
\coqdocindent{4.00em}
\coqdoctac{refine} (\coqdocnotation{(}\coqdocdefinition{ap10\_path\_arrow} \coqdocvar{\_} \coqdocvar{\_} \coqdocvar{\_} \coqdocvar{p}\coqdocnotation{)} \coqdocnotation{@} \coqdocvar{\_}).\coqdoceol
\coqdocindent{4.00em}
\coqdocvar{f\_ap}. \coqdoctac{apply} \coqdoclemma{allpath\_hprop}.\coqdoceol
\coqdocemptyline
\coqdocindent{4.00em}
\coqdoctac{intro} \coqdocvar{eq}.\coqdoceol
\coqdocindent{4.00em}
\coqdoctac{apply} \coqdocnotation{(}\coqdocdefinition{ap} \coqdocdefinition{apD10}\coqdocnotation{)\^{}-1}. \coqdoctac{apply} \coqdocdefinition{path\_forall}. \coqdoctac{intro} \coqdocvar{p}.\coqdoceol
\coqdocindent{4.00em}
\coqdoctac{refine} (\coqdocnotation{(}\coqdocdefinition{apD10\_path\_forall} \coqdocvar{\_} \coqdocvar{\_} \coqdocvar{\_} \coqdocvar{p}\coqdocnotation{)} \coqdocnotation{@} \coqdocvar{\_}).\coqdoceol
\coqdocindent{4.00em}
\coqdoctac{unfold} \coqdocmethod{O\_unit}. \coqdoctac{simpl}.\coqdoceol
\coqdocindent{4.00em}
\coqdoctac{rewrite} \coqdocdefinition{apD10\_path\_arrow}. \coqdoctac{rewrite} \coqdocdefinition{apD10\_path\_arrow}. \coqdoctac{reflexivity}.\coqdoceol
\coqdocemptyline
\coqdocindent{3.00em}
\coqdocvar{equiv\_via} (@\coqdocdefinition{O\_functor} \coqref{Ch07.Ex13.open modality}{\coqdocdefinition{open\_modality}} \coqdocvar{\_} \coqdocvar{\_} (\coqdocvar{f} \coqdocnotation{o} \coqdocvar{h}) \coqdocnotation{o} \coqdocnotation{(}\coqdocmethod{O\_unit} \coqdocvariable{P}\coqdocnotation{)}\coqdoceol
\coqdocindent{8.50em}
\coqdocnotation{=}\coqdoceol
\coqdocindent{8.50em}
\coqdocdefinition{O\_functor} (\coqdocvar{g} \coqdocnotation{o} \coqdocvar{j}) \coqdocnotation{o} \coqdocnotation{(}\coqdocmethod{O\_unit} \coqdocvariable{P}\coqdocnotation{)}).\coqdoceol
\coqdocindent{3.00em}
\coqdoctac{refine} (\coqdocdefinition{equiv\_adjointify} \coqdocvar{\_} \coqdocvar{\_} \coqdocvar{\_} \coqdocvar{\_}).\coqdoceol
\coqdocindent{4.00em}
\coqdoctac{intro} \coqdocvar{eq}. \coqdoctac{apply} \coqdocdefinition{path\_forall}. \coqdoctac{intro} \coqdocvar{p}.\coqdoceol
\coqdocindent{4.00em}
\coqdoctac{refine} (\coqdocnotation{(}\coqdocdefinition{O\_unit\_natural} \coqdocvar{\_} \coqdocvar{\_}\coqdocnotation{)} \coqdocnotation{@} \coqdocvar{\_}).\coqdoceol
\coqdocindent{4.00em}
\coqdoctac{refine} (\coqdocvar{\_} \coqdocnotation{@} \coqdocnotation{(}\coqdocdefinition{O\_unit\_natural} \coqdocvar{\_} \coqdocvar{\_}\coqdocnotation{)\^{}}).\coqdoceol
\coqdocindent{4.00em}
\coqdoctac{apply} (\coqdocdefinition{apD10} \coqdocvar{eq} \coqdocvar{p}).\coqdoceol
\coqdocemptyline
\coqdocindent{4.00em}
\coqdoctac{intro} \coqdocvar{eq}. \coqdoctac{apply} \coqdocdefinition{path\_forall}. \coqdoctac{intro} \coqdocvar{p}.\coqdoceol
\coqdocindent{4.00em}
\coqdoctac{refine} (\coqdocnotation{(}\coqdocdefinition{O\_unit\_natural} \coqdocvar{\_} \coqdocvar{\_}\coqdocnotation{)\^{}} \coqdocnotation{@} \coqdocvar{\_}).\coqdoceol
\coqdocindent{4.00em}
\coqdoctac{refine} (\coqdocvar{\_} \coqdocnotation{@} \coqdocnotation{(}\coqdocdefinition{O\_unit\_natural} \coqdocvar{\_} \coqdocvar{\_}\coqdocnotation{)}).\coqdoceol
\coqdocindent{4.00em}
\coqdoctac{apply} (\coqdocdefinition{apD10} \coqdocvar{eq} \coqdocvar{p}).\coqdoceol
\coqdocemptyline
\coqdocindent{4.00em}
\coqdoctac{intro} \coqdocvar{eq}.\coqdoceol
\coqdocindent{4.00em}
\coqdoctac{apply} \coqdocnotation{(}\coqdocdefinition{ap} \coqdocdefinition{apD10}\coqdocnotation{)\^{}-1}. \coqdoctac{apply} \coqdocdefinition{path\_forall}. \coqdoctac{intro} \coqdocvar{p}.\coqdoceol
\coqdocindent{4.00em}
\coqdoctac{refine} (\coqdocnotation{(}\coqdocdefinition{apD10\_path\_forall} \coqdocvar{\_} \coqdocvar{\_} \coqdocvar{\_} \coqdocvar{p}\coqdocnotation{)} \coqdocnotation{@} \coqdocvar{\_}).\coqdoceol
\coqdocindent{4.00em}
\coqdoctac{apply} \coqdocdefinition{moveR\_Mp}. \coqdoctac{apply} \coqdocdefinition{moveR\_pV}.\coqdoceol
\coqdocindent{4.00em}
\coqdoctac{refine} (\coqdocnotation{(}\coqdocdefinition{apD10\_path\_forall} \coqdocvar{\_} \coqdocvar{\_} \coqdocvar{\_} \coqdocvar{p}\coqdocnotation{)} \coqdocnotation{@} \coqdocvar{\_}).\coqdoceol
\coqdocindent{4.00em}
\coqdocvar{hott\_simpl}.\coqdoceol
\coqdocemptyline
\coqdocindent{4.00em}
\coqdoctac{intro} \coqdocvar{eq}.\coqdoceol
\coqdocindent{4.00em}
\coqdoctac{apply} \coqdocnotation{(}\coqdocdefinition{ap} \coqdocdefinition{apD10}\coqdocnotation{)\^{}-1}. \coqdoctac{apply} \coqdocdefinition{path\_forall}. \coqdoctac{intro} \coqdocvar{p}.\coqdoceol
\coqdocindent{4.00em}
\coqdoctac{refine} (\coqdocnotation{(}\coqdocdefinition{apD10\_path\_forall} \coqdocvar{\_} \coqdocvar{\_} \coqdocvar{\_} \coqdocvar{p}\coqdocnotation{)} \coqdocnotation{@} \coqdocvar{\_}).\coqdoceol
\coqdocindent{4.00em}
\coqdoctac{apply} \coqdocdefinition{moveR\_Vp}. \coqdoctac{apply} \coqdocdefinition{moveR\_pM}.\coqdoceol
\coqdocindent{4.00em}
\coqdoctac{refine} (\coqdocnotation{(}\coqdocdefinition{apD10\_path\_forall} \coqdocvar{\_} \coqdocvar{\_} \coqdocvar{\_} \coqdocvar{p}\coqdocnotation{)} \coqdocnotation{@} \coqdocvar{\_}).\coqdoceol
\coqdocindent{4.00em}
\coqdocvar{hott\_simpl}.\coqdoceol
\coqdocemptyline
\coqdocindent{3.00em}
\coqdocvar{equiv\_via} (@\coqdocdefinition{O\_functor} \coqref{Ch07.Ex13.open modality}{\coqdocdefinition{open\_modality}} \coqdocvar{\_} \coqdocvar{\_} (\coqdocvar{f} \coqdocnotation{o} \coqdocvar{h}) \coqdoceol
\coqdocindent{8.50em}
\coqdocnotation{=}\coqdoceol
\coqdocindent{8.50em}
\coqdocdefinition{O\_functor} (\coqdocvar{g} \coqdocnotation{o} \coqdocvar{j})).\coqdoceol
\coqdocindent{3.00em}
\coqdoctac{refine} (\coqdocdefinition{equiv\_adjointify} \coqdocvar{\_} \coqdocvar{\_} \coqdocvar{\_} \coqdocvar{\_}).\coqdoceol
\coqdocindent{4.00em}
\coqdoctac{intros} \coqdocvar{eq}.\coqdoceol
\coqdocindent{4.00em}
\coqdoctac{apply} \coqdocdefinition{path\_arrow}. \coqdoctac{intro} \coqdocvar{k}. \coqdoctac{apply} \coqdocdefinition{path\_arrow}. \coqdoctac{intro} \coqdocvar{p}.\coqdoceol
\coqdocindent{4.00em}
\coqdocvar{path\_via} ((@\coqdocdefinition{O\_functor} \coqref{Ch07.Ex13.open modality}{\coqdocdefinition{open\_modality}} \coqdocvar{\_} \coqdocvar{\_} (\coqdocvar{f} \coqdocnotation{o} \coqdocvar{h}) \coqdocnotation{o} \coqdocmethod{O\_unit} \coqdocvariable{P}) \coqdocvar{p} \coqdocvar{p}).\coqdoceol
\coqdocindent{4.00em}
\coqdoctac{unfold} \coqdocdefinition{compose}. \coqdocvar{f\_ap}. \coqdoctac{apply} \coqdoclemma{allpath\_hprop}.\coqdoceol
\coqdocindent{4.00em}
\coqdocvar{path\_via} ((@\coqdocdefinition{O\_functor} \coqref{Ch07.Ex13.open modality}{\coqdocdefinition{open\_modality}} \coqdocvar{\_} \coqdocvar{\_} (\coqdocvar{g} \coqdocnotation{o} \coqdocvar{j}) \coqdocnotation{o} \coqdocmethod{O\_unit} \coqdocvariable{P}) \coqdocvar{p} \coqdocvar{p}).\coqdoceol
\coqdocindent{4.00em}
\coqdoctac{apply} (\coqdocdefinition{ap10} (\coqdocdefinition{ap10} \coqdocvar{eq} \coqdocvar{p}) \coqdocvar{p}).\coqdoceol
\coqdocindent{4.00em}
\coqdoctac{unfold} \coqdocdefinition{compose}. \coqdocvar{f\_ap}. \coqdoctac{apply} \coqdoclemma{allpath\_hprop}.\coqdoceol
\coqdocemptyline
\coqdocindent{4.00em}
\coqdoctac{intros} \coqdocvar{eq}. \coqdoctac{apply} \coqdocdefinition{path\_arrow}. \coqdoctac{intro} \coqdocvar{p}. \coqdoctac{unfold} \coqdocdefinition{compose}. \coqdocvar{f\_ap}.\coqdoceol
\coqdocemptyline
\coqdocindent{4.00em}
\coqdoctac{intro} \coqdocvar{eq}.\coqdoceol
\coqdocindent{4.00em}
\coqdoctac{apply} \coqdocnotation{(}\coqdocdefinition{ap} \coqdocdefinition{ap10}\coqdocnotation{)\^{}-1}. \coqdoctac{apply} \coqdocdefinition{path\_forall}. \coqdoctac{intro} \coqdocvar{k}.\coqdoceol
\coqdocindent{4.00em}
\coqdoctac{refine} (\coqdocnotation{(}\coqdocdefinition{ap10\_path\_arrow} \coqdocvar{\_} \coqdocvar{\_} \coqdocvar{\_} \coqdocvar{\_}\coqdocnotation{)} \coqdocnotation{@} \coqdocvar{\_}).\coqdoceol
\coqdocindent{4.00em}
\coqdoctac{apply} \coqdocnotation{(}\coqdocdefinition{ap} \coqdocdefinition{ap10}\coqdocnotation{)\^{}-1}. \coqdoctac{apply} \coqdocdefinition{path\_forall}. \coqdoctac{intro} \coqdocvar{p}.\coqdoceol
\coqdocindent{4.00em}
\coqdoctac{refine} (\coqdocnotation{(}\coqdocdefinition{ap10\_path\_arrow} \coqdocvar{\_} \coqdocvar{\_} \coqdocvar{\_} \coqdocvar{\_}\coqdocnotation{)} \coqdocnotation{@} \coqdocvar{\_}).\coqdoceol
\coqdocindent{4.00em}
\coqdoctac{apply} \coqdocdefinition{moveR\_Mp}. \coqdoctac{apply} \coqdocdefinition{moveR\_pM}.\coqdoceol
\coqdocindent{4.00em}
\coqdocvar{path\_via} (\coqdocdefinition{ap10} (\coqdocdefinition{apD10} \coqdocvar{eq} (\coqdocmethod{O\_unit} \coqdocvariable{P} \coqdocvar{p})) \coqdocvar{p}).\coqdoceol
\coqdocindent{4.00em}
\coqdocvar{f\_ap}. \coqdoctac{refine} (\coqdocnotation{(}\coqdocdefinition{ap10\_path\_arrow} \coqdocvar{\_} \coqdocvar{\_} \coqdocvar{\_} \coqdocvar{\_}\coqdocnotation{)} \coqdocnotation{@} \coqdocvar{\_}). \coqdoctac{reflexivity}.\coqdoceol
\coqdocindent{4.00em}
\coqdoctac{unfold} \coqdocdefinition{ap10}.\coqdoceol
\coqdocindent{4.00em}
\coqdoctac{apply} \coqdocdefinition{moveL\_pV}.\coqdoceol
\coqdocindent{4.00em}
\coqdocvar{path\_via} (\coqdocdefinition{apD10} (\coqdocnotation{(}\coqdocdefinition{apD10} \coqdocvar{eq} (\coqdocmethod{O\_unit} \coqdocvariable{P} \coqdocvar{p})\coqdocnotation{)} \coqdoceol
\coqdocindent{12.50em}
\coqdocnotation{@} \coqdocnotation{(}\coqdocdefinition{ap11} 1 (\coqdoclemma{allpath\_hprop} (\coqdocmethod{O\_unit} \coqdocvariable{P} \coqdocvar{p}) \coqdocvar{k})\coqdocnotation{)}) \coqdocvar{p}).\coqdoceol
\coqdocindent{4.00em}
\coqdoctac{apply} \coqdocdefinition{inverse}. \coqdoctac{apply} (\coqdocdefinition{apD10\_pp} (\coqdocdefinition{apD10} \coqdocvar{eq} (\coqdocmethod{O\_unit} \coqdocvariable{P} \coqdocvar{p})) \coqdocvar{\_} \coqdocvar{p}).\coqdoceol
\coqdocindent{4.00em}
\coqdoctac{apply} \coqdocdefinition{moveL\_Vp}.\coqdoceol
\coqdocindent{4.00em}
\coqdoctac{refine} (\coqdocnotation{(}\coqdocdefinition{apD10\_pp} (\coqdocdefinition{ap11} 1 (\coqdoclemma{allpath\_hprop} \coqdocvar{k} (\coqdocmethod{O\_unit} \coqdocvariable{P} \coqdocvar{p}))) \coqdocvar{\_} \coqdocvar{p}\coqdocnotation{)\^{}} \coqdocnotation{@} \coqdocvar{\_}).\coqdoceol
\coqdocindent{4.00em}
\coqdocvar{f\_ap}. \coqdoctac{induction} \coqdocvar{eq}. \coqdocvar{hott\_simpl}.\coqdoceol
\coqdocindent{4.00em}
\coqdoctac{apply} \coqdocdefinition{moveR\_pM}. \coqdoctac{refine} (\coqdocvar{\_} \coqdocnotation{@} \coqdocnotation{(}\coqdocdefinition{concat\_1p} \coqdocvar{\_}\coqdocnotation{)\^{}}).\coqdoceol
\coqdocindent{4.00em}
\coqdoctac{refine} (\coqdocvar{\_} \coqdocnotation{@} \coqdocnotation{(}\coqref{Ch07.Ex13.ap11 V}{\coqdoclemma{ap11\_V}} \coqdocvar{\_} \coqdocvar{\_} \coqdocvar{\_} \coqdocvar{\_}\coqdocnotation{)}).\coqdoceol
\coqdocindent{4.00em}
\coqdocvar{f\_ap}. \coqdoctac{apply} \coqdoclemma{allpath\_hprop}.\coqdoceol
\coqdocemptyline
\coqdocindent{4.00em}
\coqdoctac{intro} \coqdocvar{eq}.\coqdoceol
\coqdocindent{4.00em}
\coqdoctac{apply} \coqdocnotation{(}\coqdocdefinition{ap} \coqdocdefinition{ap10}\coqdocnotation{)\^{}-1}. \coqdoctac{apply} \coqdocdefinition{path\_forall}. \coqdoctac{intro} \coqdocvar{p}.\coqdoceol
\coqdocindent{4.00em}
\coqdoctac{refine} (\coqdocnotation{(}\coqdocdefinition{ap10\_path\_arrow} \coqdocvar{\_} \coqdocvar{\_} \coqdocvar{\_} \coqdocvar{\_}\coqdocnotation{)} \coqdocnotation{@} \coqdocvar{\_}).\coqdoceol
\coqdocemptyline
\coqdocindent{4.00em}
\coqdocvar{admit}.\coqdoceol
\coqdocindent{4.00em}
\coqdocvar{admit}.\coqdoceol
\coqdocemptyline
\coqdocemptyline
\coqdocemptyline
\coqdocindent{3.00em}
\begin{coqdoccomment}
\coqdocindent{0.50em}
XXX\coqdocindent{0.50em}
\end{coqdoccomment}
\coqdoceol
\coqdocemptyline
\coqdocemptyline
\coqdocemptyline
\coqdocemptyline
\coqdocemptyline
\coqdocemptyline
\coqdocindent{1.00em}
\begin{coqdoccomment}
\coqdocindent{0.50em}
preserves\coqdocindent{0.50em}
products\coqdocindent{0.50em}
\end{coqdoccomment}
\coqdoceol
\coqdocindent{1.00em}
- \coqdoctac{intros} \coqdocvar{A} \coqdocvar{B}. \coqdoctac{unfold} \coqdocmethod{O}. \coqdoctac{simpl}.\coqdoceol
\coqdocindent{2.00em}
\coqdoctac{apply} \coqdoclemma{equiv\_inverse}. \coqdoctac{apply} \coqdocdefinition{equiv\_prod\_corect}.\coqdoceol
\coqdocnoindent
\coqdocvar{Admitted}.\coqdoceol
\coqdocemptyline
\coqdocnoindent
\coqdockw{End} \coqref{Ch07.Ex13.OpenModality}{\coqdocsection{OpenModality}}.\coqdoceol
\coqdocemptyline
\coqdocnoindent
\coqdockw{End} \coqref{Ch07}{\coqdocmodule{Ex13}}.\coqdoceol
\coqdocemptyline
\end{coqdoccode}
\exer{7.14}{251}  \exer{7.15}{251} \begin{coqdoccode}
\end{coqdoccode}
